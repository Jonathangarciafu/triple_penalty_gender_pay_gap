
\section{Introducción}

Ya sea desde un punto de vista normativo o empírico, la discriminación de género es perjudicial para la economía, la sociedad y, aún más importante, para el grupo mayormente afectado, las mujeres. El crecimiento y el desarrollo económico recientes han aumentado el acceso de las mujeres a la educación y la salud; sin embargo, persiste la brecha salarial de género en el mundo desarrollado y en desarrollo. La Organización Internacional del Trabajo (OIT) pronostica que, si se continúa al ritmo actual, se necesitarán hasta 270 años para cerrar dicha brecha \citep{ILO2018}. Sin embargo, mientras la brecha persista, los retornos desiguales de la educación y la productividad siguen representando pérdidas de producción y bienestar para las mujeres y sus hogares.

Recientemente, se han realizado esfuerzos sustanciales e investigaciones innovadoras para comprender las diferencias ``inexplicables'' en los ingresos por género – aquellas diferencias que no pueden ser explicadas por productividad u otras variables observables como lo es el caso de educación o experiencia. Parte de esta investigación se ha concentrado en comprender cómo la disyuntiva entre tiempo e ingresos impuesta por el proceso de crianza de los hijos afecta la brecha salarial de género. Esta literatura muestra que ambos padres enfrentan la decisión de trabajar a tiempo completo u optar por un trabajo flexible y, en muchos casos, salir del mercado laboral. La literatura también muestra que son las madres quien en su mayoría optan por trabajar menos tiempo o hasta salir de la fuerza laboral en comparación con una menor proporción de padres, generando un efecto negativo duradero en los ingresos de las madres durante su estadía en el mercado laboral. \citep{Berniell2021,Boca2013,Goldin2014,Kleven2019}. 

La mayor parte de la investigación empírica en estos temas se ha realizado en países desarrollados, especialmente en Estados Unidos y Europa, donde los estados de bienestar, las políticas de apoyo familiar y las estructuras del mercado laboral difieren sustancialmente de los del mundo en desarrollo. En América Latina, la oferta limitada de "trabajos de calidad" puede jugar un papel importante en la toma de decisiones de los padres para optar por opciones más flexibles durante la crianza de sus hijos. Los trabajos flexibles pueden ser principalmente aquellos trabajos encontrados en el sector informal, lo cual corresponden una gran parte de la oferta laboral de la región, y que en su mayoría implica trabajos sin derecho a seguro médico y/o pensiones, así como mal remunerados. Así mismo, las diferencias en las regulación de licencia de paternidad para madres y padres pueden contribuir al aumento de las brechas salariales de las madres a través de un aumento en discriminación del empleador hacia las mujeres. Por ejemplo, en Nicaragua, mientras que las madres tienen derecho a 12 o hasta 14 semanas de licencia cuando dan a luz, los padres solamente tienen derecho a cinco días según el código laboral vigente.

Nicaragua, con una alta participación laboral femenina (70 por ciento), un alto nivel de informalidad (80 por ciento) y políticas familiares como el costo compartido de la licencia parental entre empleadores (40 por ciento) y el Instituto Nacional de Seguridad Social (60 por ciento), ofrece un entorno apropiado e interesante para investigar la brecha salarial de género, así como las implicaciones de la paternidad y la informalidad. Adicionalmente, el país cuenta con una encuesta cuatrimestral sobre la fuera laboral, la Encuesta de Hogares Continua (ECH), un panel con más de 3,000 hogares, con representatividad a nivel urbano y rural que permite el estudio de dinámicas del mercado laboral, así como controlar por características no observables a nivel de individuo, que son contantes a través del tiempo, y que por ende contribuye a reducir el sesgo de variable omitida. 
 
Al identificar la existencia de una brecha salarial de género y sus cambios en diferentes grupos en el mercado laboral, este estudio tiene como objetivo proveer evidencia sobre la posible triple penalidad que las mujeres podrían enfrentar en el mercado laboral de Nicaragua debido a la maternidad y su inserción en el sector informal. Por lo tanto, en el presente articulo analizo la brecha salarial de género, primero a través de la estimación de ecuaciones de salariales à la Mincer controlando por variables a nivel individual, de hogar, del lado de la oferta y por characteristicas no observables a nivel de individuo, y segundo a través la estimación de descomposiciones Blinder-Oaxaca para diferentes submuestras (mujeres y hombres, madres y padres, y madres y padres en el sector informal) con el fin de identificar brechas salariales después de resolver por la potencial amenaza de autoselección de la muestra. 

Al estimar los modelos antes explicados encuentro evidencia de la existencia de una brecha de género entre hombres y mujeres cercana al 19\%, y que ésta brecha aumenta al comparar madres y padres en el sector formal. La brecha entre padres y madres en el sector informal parece ser menor que en el sector formal, sin embargo, este resultado es causado por la diferencias en atributos y cualidades como el nivel de educación, la cual desaparece después de controlar por selección de la muestra. Por lo tanto, la evidencia provista en el presente estudio favorece a la hipotesis de una triple penalidad. Las mujeres en Nicaragua enfrentan una penalidad adicional por su condición de madre así como por su condición de informalidad.  

Adicionalmente, investigo la posible naturaleza de dicha brecha de género al proveer evidencia de una de las potenciales explicaciones, la creencia de que las mujeres, y especialmente las madres, prefieren trabajos flexibles antes de trabajos tiempo completo para así poder dedicar tiempo a las tareas reproductivas del hogar. Analizo patrones de mobilidad laboral para hombres, mujeres, padres y madres, enfocandome en identificar diferencias en sus permanencia en empleos formales o, equivalentemente, transiciones hacia empleos flexibles, en este caso representados como empleos informales. Al usar matrices de transiciones condicionales encuentro que las tasas de permanencia en empleos formales es mayor para madres comparado con hombres y padres. Así mismo, las transiciones fuera de inactividad y desempleo hacia la actividad en la fuerza laboral son mayores para madres en comparación con mujeres sin hijos, hombres y padres. Las mujeres incrementan sus probabilidad de entrar en trabajos informales (cuenta propia o asalariada) cuando se convierten en madre. 

El resto del artículo está estructurado de la siguiente manera. La sección dos ofrece una revisión de la literatura existente sobre la brecha salarial de género, su medición y los factores actuales que se cree que la explican, con un enfoque principal en la investigación que examina los factores "inexplicables". En la sección tres, explico los datos y los modelos econométricos utilizados en el estudio. En la sección cuatro, proporciono una breve descripción del mercado laboral nicaragüense con un enfoque en las mujeres. La sección cinco se presenta los principales resultados de las estimaciones y la última sección seis concluye y ofrece recomendaciones de política económica y social.
