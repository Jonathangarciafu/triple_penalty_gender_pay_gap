%-----------------------------------------------------------------
\section*{Referencias}
%-----------------------------------------------------------------
\begingroup
\noindent
\setlength{\parindent}{-0.5in}
\setlength{\leftskip}{0.5in}
%\setlength{\parskip}{10pt}

Akerlof, G. A., \& Kranton, R. E. (2000). Economics and Identity*. Quarterly Journal of Economics, 115(3), 715–753. \url{https://doi.org/10.1162/003355300554881}

Akerlof, G. A., \& Kranton, R. E. (2002). Identity and Schooling: Some Lessons for the Economics of Education. Journal of Economic Literature, 40(4), 1167–1201. \url{https://doi.org/10.1257/002205102762203585}

Akerlof, G. A., \& Kranton, R. E. (2005). Identity and the economics of organizations. Journal of Economic Perspectives, 19(1), 9–32. \url{https://doi.org/10.1257/0895330053147930}

Altonji, J. G., \& Blank, R. M. (1999a). Chapter 48 Race and gender in the labor market. Handbook of Labor Economics. \url{https://doi.org/10.1016/S1573-4463(99)30039-0}

Arcidiacono, P. (2004). Ability sorting and the returns to college major. Journal of Econometrics, 121(1–2), 343–375. \url{https://doi.org/10.1016/J.JECONOM.2003.10.010}

Arrow, K. J. (1973). The theory of discrimination. In discrimination in labor markets (pp. 3–33). Princeton University Press, Princeton. \url{https://doi.org/10.1515/9781400867066-003}

Becker, G. (1985). Human Capital, Effort, and the Sexual Division of Labor. Journal of Labor Economics, 3(1), S33-58. Retrieved from \url{https://econpapers.repec.org/article/ucpjlabec/v\_3a3\_3ay\_3a1985\_3ai\_3a1\_3ap\_3as33-58.html}

Becker, G. S. (1957). The Economics of Discrimination. University of Chicago Press, 18(3), 276. \url{https://doi.org/10.2307/3708738}

Becker, G. S., Hubbard, W. H. J., \& Murphy, K. M. (2010). The Market for College Graduates and the Worldwide Boom in Higher Education of Women. American Economic Review, 100(2), 229–233. \url{https://doi.org/10.1257/aer.100.2.229}

Ben Yahmed, S. (2018). Formal but Less Equal. Gender Wage Gaps in Formal and Informal Jobs in Urban Brazil. World Development, 101, 73–87. \url{https://doi.org/10.1016/j.worlddev.2017.08.012}

Benard, S., \& Correll, S. J. (2010). Normative discrimination and the motherhood penalty. Gender and Society, 24(5), 616–646. \url{https://doi.org/10.1177/0891243210383142}

Berniell, L., Mata, D. De, Edo, M., \& Marchionni, M. (2019). Gender Gaps in Labor Informality : The Motherhood Effect. Documentos de Trabajo Del CEDLAS, no. 247(Issue 247), 1–41. Retrieved from \url{http://sedici.unlp.edu.ar/handle/10915/77486}

Bertrand, M. (2011). New perspectives on gender. Handbook of Labor Economics (Vol. 4). Elsevier. \url{https://doi.org/10.1016/S0169-7218(11)02415-4}

Bian, L., Leslie, S.-J., \& Cimpian, A. (2017). Gender stereotypes about intellectual ability emerge early and influence children's interests. Science (New York, N.Y.), 355(6323), 389–391. \url{https://doi.org/10.1126/science.aah6524}

Blau, F. D., \& Kahn, L. (1995). The Gender Earnings Gap: Some International Evidence. NBER Chapters. Cambridge, MA. \url{https://doi.org/10.3386/w4224}

Blau, F. D., \& Kahn, L. M. (2000). Gender Differences in Pay. Journal of Economic Perspectives, 14(4), 75–100. \url{https://doi.org/10.1257/jep.14.4.75}

Blau, F. D., \& Kahn, L. M. (2007). The Gender Pay Gap. Academy of Management Perspectives, 21(1), 7–23. \url{https://doi.org/10.5465/amp.2007.24286161}

Blau, F. D., \& Kahn, L. M. (2017). The Gender Wage Gap: Extent, Trends, and Explanations. Journal of Economic Literature, 55(3), 789–865. \url{https://doi.org/10.1257/jel.20160995}

Blinder, A. S. (1973). Wage Discrimination: Reduced Form and Structural Estimates. The Journal of Human Resources, 8(4), 436. \url{https://doi.org/10.2307/144855}

Borjas, G. J. (2016). Labor Economics. McGraw-hill (Vol. 7).

Borrowman, M., \& Klasen, S. (2019). Drivers of Gendered Sectoral and Occupational Segregation in Developing Countries. Feminist Economics, 1–33. \url{https://doi.org/10.1080/13545701.2019.1649708}

Bosch, M., \& Maloney, W. F. (2010). Comparative analysis of labor market dynamics using Markov processes: An application to informality. Labour Economics, 17(4), 621–631. \url{https://doi.org/10.1016/j.labeco.2010.01.005}

Ceci, S. J., Ginther, D. K., Kahn, S., \& Williams, W. M. (2014). Women in academic science: A changing landscape. Psychological Science in the Public Interest, Supplement, 15(3), 75–141. \url{https://doi.org/10.1177/1529100614541236}

Cha, Y., \& Weeden, K. A. (2014). Overwork and the Slow Convergence in the Gender Gap in Wages. American Sociological Review, 79(3), 457–484. \url{https://doi.org/10.1177/0003122414528936}

Charles, K. K., Guryan, J., \& Pan, J. (2018). The Effects of Sexism on American Women: The Role of Norms vs. Discrimination. SSRN Electronic Journal. \url{https://doi.org/10.2139/ssrn.3233788}

Charness, G., \& Gneezy, U. (2012). Strong Evidence for Gender Differences in Risk Taking. Journal of Economic Behavior and Organization, 83(1), 50–58. \url{https://doi.org/10.1016/j.jebo.2011.06.007}

Clark, S., Kabiru, C. W., Laszlo, S., \& Muthuri, S. (2019). The Impact of Childcare on Poor Urban Women's Economic Empowerment in Africa. Demography, 56(4), 1247–1272. \url{https://doi.org/10.1007/s13524-019-00793-3}

Claudia Goldin. (2008). Gender Gap. Retrieved November 15, 2019, from \url{https://www.econlib.org/library/Enc1/GenderGap.html}

Correll, S. J., Benard, S., \& Paik, I. (2007). Getting a job: Is there a motherhood penalty? American Journal of Sociology, 112(5), 1297–1338. \url{https://doi.org/10.1086/511799}

Croson, R., \& Gneezy, U. (2009). Gender Differences in Preferences. Journal of Economic Literature, 47(2), 448–474. \url{https://doi.org/10.1257/jel.47.2.448}

Del Boca, D., Flinn, C., \& Wiswall, M. (2014). Household choices and child development. Review of Economic Studies, 81(1), 137–185. \url{https://doi.org/10.1093/restud/rdt026}

ECLAC. (2016). Women: The Most Harmed by Unemployment. Notes for Equality (Vol. 22). Retrieved from \url{http://www.cepal.org/en/publications/type/preliminary-overview-economies-latin-america-and-caribbean}

England, M. J. B. and P. (2001). The Wage Penalty for Motherhood. American Sociological Review, 53(9), 1689–1699. \url{https://doi.org/10.1017/CBO9781107415324.004}

England, P., Farkas, G., Kilbourne, B. S., \& Dou, T. (1988). Explaining Occupational Sex Segregation and Wages: Findings from a Model with Fixed Effects. American Sociological Review, 53(4), 544.\url{https://doi.org/10.2307/2095848}

Fortin, N., Lemieux, T., \& Firpo, S. (2011). Decomposition Methods in Economics. Handbook of Labor Economics (Vol. 4). \url{https://doi.org/10.1016/S0169-7218(11)00407-2}

Fortin, N. M. (2005). Gender role attitudes and the labour-market outcomes of women across OECD countries. Oxford Review of Economic Policy, 21(3), 416–438. \url{https://doi.org/10.1093/oxrep/gri024}

Fuller, W. C., Manski, C. F., \& Wise, D. A. (1982). New Evidence on the Economic Determinants of Postsecondary Schooling Choices. The Journal of Human Resources, 17(4), 477. \url{https://doi.org/10.2307/145612}

Gaddis, I., \& Klasen, S. (2014). Economic development, structural change, and women's labor force participation:: A reexamination of the feminization U hypothesis. Journal of Population Economics, 27(3), 639–681. \url{https://doi.org/10.1007/s00148-013-0488-2}

Gneezy, U., Leonard, K. L., \& List, J. A. (2009). Gender Differences in Competition: Evidence From a Matrilineal and a Patriarchal Society. Econometrica, 77(5), 1637–1664. \url{https://doi.org/10.3982/ecta6690}

Goldin, C. (2006a). The Quiet Revolution That Transformed Women's Employment, Education, and Family. American Economic Review, 96(2), 1–21. \url{https://doi.org/10.1257/000282806777212350}

Goldin, C. (2006b). The quiet revolution that transformed women's employment, education, and family. American Economic Review, 96(2), 1–21. \url{https://doi.org/10.1257/000282806777212350}

Goldin, C. (2014). A grand gender convergence: Its last chapter. American Economic Review, 104(4), 1091–1119. \url{https://doi.org/10.1257/aer.104.4.1091}

Goldin, C., \& Katz, L. F. (2002). The power of the pill: Oral contraceptives and women's career and marriage decisions. Journal of Political Economy, 110(4), 730–770. \url{https://doi.org/10.1086/340778}

González, K. P., Stoner, C., \& Jovel, J. E. (2003). Examining the Role of Social Capital in Access to College for Latinas: Toward a College Opportunity Framework. Journal of Hispanic Higher Education, 2(2), 146–170. \url{https://doi.org/10.1177/1538192702250620}

Heckman, J. J. (1979). Sample Selection Bias as a Specification Error. Econometrica, 47(1), 153. \url{https://doi.org/10.2307/1912352}

Hersch, J., \& Stratton, L. S. (2002). Housework and wages. Journal of Human Resources, 37(1), 217–229. \url{https://doi.org/10.2307/3069609}

ILO. (2019). A quantum leap for gender equality : for a better future of work for all. Geneva. Retrieved from \url{www.ilo.org/publns}.

Jann, B. (2008). The Blinder-Oaxaca decomposition for linear regression models. Stata Journal, 8(4), 453–479. \url{https://doi.org/10.1177/1536867x0800800402}

Klasen, S., \& Pieters, J. (2015). What explains the stagnation of female labor force participation in Urban India? World Bank Economic Review, 29(3), 449–478. \url{https://doi.org/10.1093/wber/lhv003}

Kleven, H., Landais, C., \& Søgaard, J. E. (2019). Children and Gender Inequality: Evidence from Denmark. American Economic Journal: Applied Economics, 11(4), 181–209. \url{https://doi.org/10.1257/app.20180010}

Macpherson, D. A., \& Hirsch, B. T. (1995). Wages and Gender Composition: Why do Women's Jobs Pay Less? Journal of Labor Economics, 13(3), 426–471. \url{https://doi.org/10.1086/298381}

Mincer, Jacob and Polachek, S. (1974). Family Investments in Human Capital: Earnings of Women: Comment. Journal of Political Economy, 82(2, Part 2), S109–S110. \url{https://doi.org/10.1086/260294}

Monroy, E. (2008). Equidad de Genero en el Mercado Laboral para Nicaragua. Serie de Cuadernos de Género Para …, 8(2), 1307–1319. Retrieved from \url{http://web.worldbank.org/archive/website01404/WEB/IMAGES/CGANICAR.PDF}

Niederle, M., \& Vesterlund, L. (2007, August 1). Do women shy away from competition? Do men compete too much? Quarterly Journal of Economics. Narnia. \url{https://doi.org/10.1162/qjec.122.3.1067}

Nordman, C. J., Rakotomanana, F., \& Roubaud, F. (2016). Informal versus Formal: A Panel Data Analysis of Earnings Gaps in Madagascar. World Development, 86, 1–17. \url{https://doi.org/10.1016/j.worlddev.2016.05.006}

Oaxaca, R. (1973). Male-Female Wage Differentials in Urban Labor Markets. International Economic Review, 14(3), 693. \url{https://doi.org/10.2307/2525981}

Phelps, E. S. (1972). The Statistical theory of Racism and Sexism. American Economic Review, 62(4), 659–661. \url{https://doi.org/10.2307/1806107}

Reskin, B. (1993). Sex segregation in the workplace. Annual Review of Sociology. Vol. 19, 241–270. \url{https://doi.org/10.1146/annurev.soc.19.1.241}

Ruhm, C. J. (1998). The Economic Consequences of Parental Leave Mandates: Lessons from Europe. The Quarterly Journal of Economics, 113(1), 285–317. \url{https://doi.org/10.1162/003355398555586}

Ruzik, A., \& Rokicka, M. (2012). The Gender Pay Gap in Informal Employment in Poland. SSRN Electronic Journal. \url{https://doi.org/10.2139/ssrn.1674939}

Sahoo, S., \& Klasen, S. (2018). Gender Segregation in Education and Its Implications for Labour Market Outcomes: Evidence from India. Educational Research Journal, (11660), 1–52. Retrieved from \url{https://www.iza.org/publications/dp/11660/gender-segregation-in-education-and-its-implications-for-labour-market-outcomes-evidence-from-india}

Schneeweis, N., \& Zweimüller, M. (2012). Girls, girls, girls: Gender composition and female school choice. Economics of Education Review, 31(4), 482–500. \url{https://doi.org/10.1016/j.econedurev.2011.11.002}

Sigle-Rushton, W., \& Waldfogel, J. (2007). The incomes of families with children: A cross-national comparison. Journal of European Social Policy, 17(4), 299–318. \url{https://doi.org/10.1177/0958928707082474}

Sorensen, E. (1990). The Crowding Hypothesis and Comparable Worth. The Journal of Human Resources, 25(1), 55. \url{https://doi.org/10.2307/145727}

Tansel, A. (2005). Wage Earners, Self-Employed and Gender in the Informal Sector in Turkey. SSRN Electronic Journal. \url{https://doi.org/10.2139/ssrn.263275}

The International Labour Organization. (2018). Global wage report 2018 / 19 what lies behind gender pay gaps. Retrieved from \url{https://www.ilo.org/wcmsp5/groups/public/---dgreports/---dcomm/---publ/documents/publication/wcms\_650553.pdf}

UNDP. (2014). El mercado laboral de Nicaragua desde un enfoque de género.

Waldfogel, J. (1998). Understanding the "Family Gap" in Pay for Women with Children. Journal of Economic Perspectives, 12(1), 137–156. \url{https://doi.org/10.1257/jep.12.1.137}

Weichselbaumer, D., \& Winter-Ebmer, R. (2005). A meta-analysis of the international gender wage gap. Journal of Economic Surveys, 19(3), 479–511. \url{https://doi.org/10.1111/j.0950-0804.2005.00256.x}

Williams, W. M., \& Ceci, S. J. (2015). National hiring experiments reveal 2:1 faculty preference for women on STEM tenure track. Proceedings of the National Academy of Sciences of the United States of America, 112(17), 5360–5365. \url{https://doi.org/10.1073/pnas.1418878112}

World Bank. (2011). World Development Report 2012.

World Economic Forum. (2016). The global gender gap report 2016 Insight Report. World Economic Forum (Vol. 25). \url{https://doi.org/10.1177/0192513X04267098}

\endgroup