\section{Literature review}

La brecha de género se refiere a las diferencias sistemáticas que hombres y mujeres, idénticos en cuanto a su capacidad productiva, enfrentan en el mercado laboral debido a características no productivas. Estas diferencias aparecen en los porcentajes de hombres y mujeres en la fuerza laboral, los tipos de ocupaciones que eligen y la diferencia en las remuneraciones de hombres y mujeres \citep{Goldin2008}.

El diferencial salarial total entre hombres y mujeres se puede descomponer en una parte explicada por las diferencias en las características productivas y un componente residual inexplicable. El papel de las diferencias de género en este residuo inexplicable es a menudo se denomina efecto de discriminación. La investigación empírica sobre las brechas salariales de género se ha centrado tradicionalmente en el rol de los factores específicos de género, en particular las diferencias en las calificaciones y las diferencias en el trato a trabajadores masculinos y femeninos igualmente calificados (es decir, discriminación en el mercado laboral) \citep{Blau1992,Weichselbaumer2005}

El análisis económico de la discriminación se remonta al trabajo inicial sobre la economía de la discriminación de \citet{Becker1985} y la vasta literatura que siguió. La teoría de Becker sobre la discriminación en el mercado laboral se basa en el concepto de discriminación basada en el gusto. Aplicado a las diferencias de género, este concepto se refiere a la noción de prejuicio de género y resultados económicos. Becker monetiza este prejuicio y determina sus diferentes fuentes, clasificadas como discriminación del empleador, discriminación del empleado y discriminación del cliente \citep{Borjas2016}.

Además, o incluso en ausencia de discriminación basada en prejuicios, las diferencias en las habilidades o la productividad son atribuidas a diferentes a géneros por parte del empleador. Los empleadores tienen información imperfecta sobre la productividad individual y solo conocen la productividad promedio a nivel de grupo, por ejemplo, las mujeres. Así, la productividad asumida para un trabajador por parte del empleado da lugar a diferencias inexplicables. La noción de esta teoría se basa en el trabajo inicial de \citet{Arrow1973} y \citet{Phelps1972} citado por \citet{Altonji1999} y es conocida como discriminación estadística.

La brecha salarial de género, y los factores que la determinan, han estado sujetos a cambios constantes a lo largo del tiempo, a un ritmo similar al que cambia el mercado laboral. Hasta la fecha, los factores tradicionales como la afiliación sindical, la raza, el estatus migratorio o la religión, y especialmente las variables relacionadas con el capital humano, es decir, el nivel educativo y la experiencia, explican muy poco de los diferenciales de ingreso entre hombres y mujeres \citep{Blau2017,Weichselbaumer2005}. Recientemente, la literatura ha analizado cada vez más otros factores debido a nuevos fenómenos observados en el mercado laboral o a las mejoras en los métodos utilizados para estudiar las contribuciones de estos factores a las diferencias salariales.

A lo largo de las mejoras del siglo pasado, la brecha de género en el acceso a la educación casi ha desaparecido y la del nivel de educación incluso se ha revertido en países desarrollados y en desarrollo \citep{Becker2010,Gaddis2013,Goldin2006}.  Esta tendencia también se observa en el caso del nivel de experiencia en el mercado laboral. En los Estados Unidos, en 2011, las mujeres tenían un tres por ciento más de probabilidades de obtener un título avanzado que los hombres, y los años de experiencia a tiempo completo de los hombres (17.8 años) eran solo 1.4 años más altos que los de las mujeres \citep{Blau2017}. Así, en lugar de explicar las diferencias salariales, el nivel de educación y la experiencia actualmente representan más bien un factor reductor de dicha brecha.

Hasta el presente, el foco de atención de la evidencia empírica en género y el mercado laboral se ha desplazado hacia el análisis de tendencias y fenómenos recientes, como el estancamiento de la participación laboral femenina o el aumento de la liberalización comercial y su impacto en las mujeres. Asimismo, se han realizado importantes esfuerzos e innovaciones con el objetivo de esclarecer la parte inexplicable de la brecha salarial entre hombres y mujeres y los factores que la componen, incluyendo la formación de la familia y los factores de procreación y crianza.

%-----------------------------------------------------------------
\subsection{Interrupciones del mercado laboral}
%-----------------------------------------------------------------
Las diferencias en la permanencia en el mercado laboral han estado en el centro del debate como posibles factores determinantes de la brecha salarial de género \citep{Mincer1974}. Las interrupciones en el mercado laboral de las mujeres, principalmente debido a la formación de una familia, es decir, el matrimonio y/o el parto, todavía contribuyen significativamente a las diferencias salariales entre hombres y mujeres \citep{WorldBank2012}. Por ejemplo, \citet{Goldin2014} encontró que las interrupciones laborales para los profesionales de MBA y finanzas en los Estados Unidos contribuyen hasta un 30 por ciento de la brecha de género en los ingresos.

Aunque la familia, su estructura y funcionamiento han cambiado con el tiempo, aún se observan diferencias entre países y regiones derivadas principalmente de las instituciones formales e informales que prevalecen en sus países. En Argentina, Brasil, Ghana, México, Serbia y Tailandia, las diferencias entre hombres y mujeres en el uso del tiempo para el trabajo reproductivo afectan negativamente su probabilidad de transitar hacia "buenos trabajos" y aumentan sus probabilidades de estar en el sector informal de trabajadores autónomos o la fuerza laboral inactiva \citep{Bosch2010}.

La magnitud de la contribución de las interrupciones laborales depende de su duración y como son cuantificadas. La evidencia empírica soporta la hipótesis de heterogeneidad en el impacto de interrupciones en los salarios de las mujeres dependiendo de su duración. Interrupciones cortas están generalmente asociadas con impactos positivos, mientras que interrupciones largas con efectos negativos \citep{Blau2000,Ruhm1998,Waldfogel1998}. No obstante, la mayor parte de la literatura empírica se encuentra con el problema de la carencia de una forma adecuada de medir las interrupciones laborales y su duración, lo que conduce a utilizar sustitutos imperfectos. \citet{Nordman2016} explotan una nueva base de datos de Madagascar que les permitió comparar la contribución de la experiencia real y potencial (proxy) a la brecha salariar de género proporcionando evidencia de que el uso de proxis como experiencia laboral subestima la contribución de las interrupciones del trabajo a las diferencias salariales de género.

La experiencia adquirida en el trabajo tiene una contribución significativa a la brecha de ingresos entre hombres y mujeres a lo largo de sus carreras profesionales \citep{Kleven2019}.  Dichas diferencias presentan magnitudes heterogéneas entre regiones, representando desde un 1 por ciento de los determinantes del salario para las mujeres en Europa y hasta un 10 por ciento en los países en desarrollo \citep{Weichselbaumer2005}. 
%-----------------------------------------------------------------
\subsection{Penalidad por maternidad}                        
%-----------------------------------------------------------------
La evidencia empírica sugiere que, además del impacto de la permanencia en el mercado laboral, los roles de género y las decisiones sobre la formación de la familia podrían afectar los salarios relativos de las mujeres \citep{Blau2017}. Este fenómeno se denominó inicialmente brecha familiar, y representa la diferencia salarial entre mujeres con hijos y mujeres sin hijos \citep{Waldfogel1998}. Actualmente, es más conocido en la literatura como la penalidad de maternidad.

\citet{Budig2001} utilizan un modelo de efectos fijos para Estados Unidos con la finalidad de medir la magnitud de la penalidad de maternidad y obtuvieron resultados que muestran una penalización salarial del 7 por ciento por hijo. También encontraron que la penalidad es mayor para las mujeres casadas que para las solteras y que las mujeres con (más) hijos tienen menos años de experiencia laboral. Así mismo, observaron que, después de controlar por experiencia, se mantiene la penalidad a un nivel cercano al 5 por ciento por hijo.

Utilizando datos de siete países con diferentes estados de bienestar, \citet{SigleRushton2007} encontraron que las brechas en el ingreso familiar son menores en los países nórdicos, comparado a los angloamericanos y ambas menores que la observadas en los países de Europa continental, concluyendo, por lo tanto, que el efecto de la maternidad difiere según el estado de bienestar.

\citet{Kleven2019} utilizan datos daneses y encuentran que la llegada de un niño crea una brecha de género en los ingresos de alrededor del 20 por ciento, causada principalmente por interrupciones del trabajo, horas de trabajo y niveles de remuneración, y que estas afecta a las madres, pero no a los padres. Las brechas de género una vez ocurren son muy estables en el tiempo y las mujeres no muestran signos de recuperación en el mercado laboral incluso diez años después del primer hijo. \citet{Kleven2019} también encontró que la penalidad por cada hijo se transmite de generación en generación, a las hijas y no a los hijos, a través de la influencia que el entorno tiene en la formación de preferencias de las niñas sobre la familia y la carrera.

Estimar y establecer un efecto causal de la pena de maternidad sigue siendo controversial debido a factores como la autoselección de las mujeres, es decir, las mujeres con salarios más bajos tienen menores costos de hijos y viceversa. Varios factores surgen de la literatura como una posible explicación de la pena de maternidad, entre ellos, la discriminación de los empleadores y la disyuntiva entre horas de trabajo y salarios más altos o trabajos con horario favorables para las madres \citep{Budig2001}.

La evidencia experimental en el ámbito de discriminación ha tenido como objetivo evaluar la existencia de tal y sus posibles determinantes. \citet{Correll2007} llevaron a cabo un experimento de laboratorio en el que se evaluaron los currículums de solicitantes de empleo del mismo sexo igualmente calificados, que solo se diferenciaban por su estado parental. Los autores observaron que la recomendación salarial fue menor para madres, mientras que los padres no fueron penalizados. Además, \citet{Benard2010} examinaron si las madres enfrentan discriminación al momento de ser evaluadas en el mercado laboral, aún al brindar evidencia indiscutible de que son competentes y están comprometidas con el trabajo remunerado. Sus resultados sugieren que las mujeres evaluadoras aplican "discriminación normativa", encontrando a las madres menos cálidas, menos agradables y más hostiles interpersonalmente que las trabajadoras similares que no son madres. Al contrario, \citet{Williams2015} no encontraron evidencia de discriminación en el mercado académico y concluyeron que la discriminación hacia las madres podría resultar de discriminación estadística debido a la percepción de los empleadores de las diferencias de productividad entre madres y no madres.

Las diferencias de género en el uso del tiempo, particularmente en actividades no remuneradas, impactan en los resultados que las mujeres obtienen en el mercado laboral de varias maneras \citep{Becker1985,Blau2017}. Por ejemplo, las largas horas que las mujeres casadas o las madres dedican a actividades reproductivas podrían reducir el esfuerzo que dedican a sus trabajos en el mercado. En esa misma línea, \citet{Hersch2002} encuentran que las tareas domésticas, especialmente las tareas de rutina diaria como cocinar y limpiar, perjudican los salarios independientemente del estado civil en los Estados Unidos. Además, los autores proporcionan evidencia de que el control del tiempo de trabajo doméstico aumenta el componente explicado de la brecha salarial de género en 14 puntos porcentuales. \citet{Cha2014} examinan el papel de un aumento en la prevalencia de largas horas de trabajo y la brecha salarial de género durante el período 1979-2009, encontrando que la diferencia en las horas trabajadas aumentó la brecha salarial de género en aproximadamente un 10 por ciento del cambio total durante este período.
%-----------------------------------------------------------------
\subsection{Informalidad}
%-----------------------------------------------------------------
Los mercados laborales de los países en desarrollo se caracterizan por altos niveles de informalidad. Por ejemplo, en América Latina, el llamado sector informal comprende una gran proporción (30 a 70 por ciento) de la fuerza laboral ocupada \citep{Maloney2004}. Los trabajos en el sector informal difieren de los trabajos formales en dimensiones como una protección social, menores ingresos laborales, menores perspectivas de carrera, más flexibilidad y jornadas laborales más cortas \citep{Berniell2021}. Por ende, las principales conclusiones de la vasta literatura sobre la brecha salarial de género que se han centrado principalmente en Estados Unidos y Europa podrían tener diferentes aplicaciones en países con altas tasas de informalidad.

\citet{Tansel2001} utilizó la descomposición de Oaxaca-Blinder para estimar las diferencias salariales entre hombres y mujeres según el diferente estado de cobertura del seguro social en Turquía. En este contexto, el autor concluyó que, en el sector con cobertura de seguridad social, los salarios de los hombres son aproximadamente dos veces más altos que los de las mujeres. Para los trabajadores asalariados fuera de la seguridad social, los salarios de los hombres están casi a la par con los de las mujeres. Estos resultados sugieren una segmentación para los hombres en el sector formal e informal y una discriminación sustancial para las mujeres en el sector privado con cobertura de seguridad social.

De manera similar, \citet{Ruzik2010} analizaron la brecha salarial de género en el sector informal de Polonia utilizando regresiones por cuantiles, en este caso analizando trabajadores formalmente registrados y no registrados, encontrando que la desigualdad de ingresos entre mujeres y hombres no registrados es más pronunciada en la cola inferior de la distribución de ganancias. En el caso de los empleados formales, la desigualdad en la parte superior de la distribución tiende a ser mayor, lo que confirma la existencia de un 'techo de cristal'. El estudio también propone que una posible explicación de los resultados es la falta de regulaciones de salario mínimo en el mercado informal y la mayor flexibilidad al momento de negociar y decidir sobre salarios en los cuantiles superiores.

Por el contrario, \citet{Yahmed2018} estudió cómo la desigualdad de género difiere entre trabajadores formales e informales en Brasil, y encontró que la brecha salarial de género bruta es aproximadamente la misma en trabajos informales y trabajos formales. Sin embargo, también encuentra que este resultado es provocado por la diferencia en los procesos de selección masculina y femenina en cada uno de los sectores. Por lo tanto, después de controlar las características observables, la brecha salarial de género ajustada es, en promedio, alrededor del 24 por ciento entre los empleados formales y alrededor del 20 por ciento entre los empleados informales.

Por último, \citet{Berniell2021} investigan más a fondo el efecto de las diferencias de género por estado de formalización y el efecto de la maternidad para Chile y otros países en desarrollo de la Organización para la Cooperación y el Desarrollo Económico (OCDE), encontrando que la maternidad produce una reducción considerable en los ingresos laborales de las madres chilenas y que esta reducción es asociada a una menor participación en la fuerza laboral y una caída en el empleo formal. Asimismo, el estudio encuentra que la penalidad por maternidad en Chile es menor que en Estados Unidos, pero es mayor que en Dinamarca.
%-----------------------------------------------------------------
\subsection{Medición}
%-----------------------------------------------------------------
Diferentes métodos han sido desarrollados y utilizados para analizar la discriminación empíricamente. La forma más común de analizar la discriminación basada en el género es comparar los ingresos masculinos y femeninos manteniendo la productividad constante. Otra manera de hacerlo sería incluir una variable dummy representando el género del individuo en un modelo de regresión salarial únicamente. Sin embargo, el procedimiento estándar para investigar las diferencias en los salarios es mediante el uso de métodos de descomposición \citep{Fortin2011,Weichselbaumer2005}.

Varios métodos de descomposición se han desarrollado desde el trabajo seminal de \citet{Oaxaca1973} y \citet{Blinder1973}. Sin embargo, la denominada descomposición Blinder-Oaxaca sigue siendo el enfoque más utilizado a la fecha. Este procedimiento permite descomponer el diferencial salarial entre hombres y mujeres en una parte explicada debido a diferencias en las características y un residuo inexplicable \citep{Fortin2011,Weichselbaumer2005}.
%-----------------------------------------------------------------
\subsection{La Brecha Salarial De Género En Nicaragua}
%-----------------------------------------------------------------
Pocos estudios han analizado la brecha salarial de género en Nicaragua. \citet{Weichselbaumer2005} realizan un meta-análisis y concluyen que una parte sustancial de la brecha salarial total puede atribuirse a diferencias en el capital humano, debido sus hallazgos que indican que las mujeres tienen diferencias en características socioeconómicas que se resultan en una menor productividad comparado con los hombres. Asimismo, \citet{Monroy2008} estima las diferencias brutas en los ingresos de los trabajadores y trabajadoras por ocupación, encontrando una diferencia promedio del 20 por ciento. Adicionalment, después de estimar las diferencias brutas en los ingresos por estatus de formalización, encuentra que la brecha en el sector informal (18 por ciento) es mayor que en el sector formal (7 por ciento).

De manera similar, el informe del \citet{PNUD2014} sobre el mercado laboral nicaragüense desde una perspectiva de género utiliza descomposiciones de Blinder-Oaxaca y encuentra que el ingreso mensual de los hombres es mayor que el de las mujeres en más del 30 por ciento tanto para las áreas rurales como urbanas. No obstante, la magnitud de coeficientes encontrados podría atribuirse a problemas de autoselección.

El presente artículo contribuye a la literatura sobre la brecha salarial de género en Nicaragua mediante la implementación de técnicas econométricas mejoradas y al aplicarlo en un conjunto de datos de panel para estimar las diferencias salariales mientras se controlan las características individuales observables y no observables, variables a nivel de hogar y posibles problemas de autoselección. Además, la característica del panel de la encuesta permite el estudio de las interrupciones laborales en los salarios y el análisis de los potenciales canales de la penalización por maternidad, en particular la probable selección de madres en trabajos más flexibles o informales. Por último, el analizar la brecha salarial de género después de resolver la autoselección podría proveer evidencia sobre el efecto particular de la licencia de maternidad con costos compartidos al comparar los sectores formales e informales. Esta política tiene el potencial de inducir prejuicios y discriminación por parte de los empleadores, y por ende el estudio de sus implicaciones es del interés de los hacedores de política. 
