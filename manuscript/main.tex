\documentclass[a4paper]{article}
%-------------------------------------------------------------------
% Document specifics
\usepackage{setspace}												% For spacing, e.g., double line spacing, line spread
\usepackage{enumitem}												% Controls layout of itemize, enumerate, description
\usepackage{footnote}												% Improve on LaTeX's footnote handling
%\usepackage[margin=1in]{geometry}									% Flexible and complete interface to document dimensions
\usepackage{fullpage}												% Set all page margins to 1.5cm
\usepackage{changepage}												% Commands to change the page layout in the middle of a document
\usepackage[table]{xcolor}											% Provides easy driver-independent access to several kinds of color 
\definecolor{airforceblue}{rgb}{0.36, 0.54, 0.66}
\definecolor{olivine}{rgb}{0.6, 0.73, 0.45}
\usepackage{colortbl}												% Allows rows and columns to be coloured
\usepackage{nameref}												% Defines a \nameref command, that makes reference to an object
\usepackage{titling}												% Provides control over the typesetting of the \maketitle command 
\usepackage[titletoc, title]{appendix}								% For appendices
%-------------------------------------------------------------------
% Tables and numbers
\usepackage[flushleft]{threeparttable}								% Tables with captions and notes with the same width
\usepackage{threeparttablex}										% Minor improvements over threeparttable
\usepackage{dcolumn}												% Align on the decimal point of numbers in tabular columns
\usepackage{multirow}												% Create tabular cells spanning multiple rows
\usepackage{longtable}												% Tables that continue to the next page
\usepackage{booktabs}												% Enhance quality of tables
\usepackage{tabularx}												% Tabulars with adjustable-width columns
\usepackage{mathrsfs, amsfonts, amssymb, amsmath}					% RSFS fonts in maths, Set of fonts for use in mathematics
%\usepackage{slashbox}												% Produces tabular cells with diagonal lines
\usepackage{siunitx}												% Set of tools for authors to typeset quantities in a consistent way
\usepackage{array}													% Extending the array and tabular environments
\usepackage[sc]{mathpazo}											% Fonts to typeset mathematics to match Palatino
%-------------------------------------------------------------------
% Floats in general
\usepackage{adjustbox}												% Provides several macros to adjust boxed content
\usepackage{graphicx}												% Provides a key-value interface for \includegraphics
\usepackage{float}													% Improves interface for floating objects
\usepackage[skip = 0pt]{caption}									% Customizes the captions in floating environments
\captionsetup{justification=centering}								% Captions setup
\usepackage{subcaption}												% Supports for sub-captions
\usepackage{rotating}												% Rotation tools, including rotated full-page floats
\usepackage{pdflscape}												% Makes landscape pages display as landscape
\usepackage{lscape}													% Place selected parts of a document in landscape
\usepackage{afterpage}												% Executes command after the next page break
\usepackage{tikz}													% For complex graphs in LaTeX
\usetikzlibrary{positioning}
%\tikzset{mynode/.style={draw,text width=1in,align=center}}
\tikzset{mynode/.style={draw,align=center}}
%\usepackage[nolists,heads]{endfloat}								% Moves floats to the end
%\usepackage[all]{hypcap}											% Adjusts the anchors of captions
%\usepackage{ctable}												% Flexible typesetting of table and figure floats using key/value directives
%-------------------------------------------------------------------
% Languages, encoding 
\usepackage[english, spanish]{babel}								% Multilingual support for Plain TeX or LaTeX
\usepackage[utf8]{inputenc}											% Accept different input encodings
\usepackage[autostyle, english = american]{csquotes}				% Provides advanced facilities for inline and display quotations
\usepackage[T1]{fontenc}											% Standard package for selecting font encodings
%-------------------------------------------------------------------
% MISC
\usepackage{lipsum}													% Easy access to the Lorem Ipsum dummy text.
\usepackage[plainpages=false,pdfpagelabels]{hyperref}				% Handles cross-referencing commands in LaTeX
\usepackage{url}													% Verbatim with URL-sensitive line breaks
\urlstyle{rm}
%-------------------------------------------------------------------
% References
%\usepackage{apacite}												% APA references thingy
\usepackage[natbib, style=apa]{biblatex}							% For better citation
\addbibresource{gender.bib} 										% Replace this with your actual bibfile name
%-------------------------------------------------------------------
% Other document setups
%%% Misplaced noalign
\makeatletter\let\expandableinput\@@input\makeatother
%%% Quotes multilingual 
\MakeOuterQuote{"}
%%% Matrices
\setcounter{MaxMatrixCols}{10}
%%% New theorem environments
\newtheorem{acknowledgement}{Acknowledgement}
\newtheorem{algorithm}{Algorithm}
\newtheorem{axiom}{Axiom}
\newtheorem{case}{Case}
\newtheorem{claim}{Claim}
\newtheorem{conclusion}{Conclusion}
\newtheorem{condition}{Condition}
\newtheorem{conjecture}{Conjecture}
\newtheorem{corollary}{Corollary}
\newtheorem{criterion}{Criterion}
\newtheorem{definition}{Definition}
\newtheorem{example}{Example}
\newtheorem{exercise}{Exercise}
\newtheorem{lemma}{Lemma}
\newtheorem{notation}{Notation}
\newtheorem{problem}{Problem}
\newtheorem{proposition}{Proposition}
\newtheorem{remark}{Remark}
\newtheorem{solution}{Solution}
\newtheorem{assumption}{Assumption}
%%% New environment for landscape tables    
\newenvironment{ltable}{\begin{landscape}\begin{table}}{\end{table}\end{landscape}}
\newenvironment{ltablelong}{\begin{landscape}\begin{longtable}}{\end{longtable}\end{landscape}}
%%% New environment for column type (hide columns)
\newcolumntype{H}{>{\setbox0=\hbox\bgroup}c<{\egroup}@{}}
\newcolumntype{P}[1]{>{\centering\arraybackslash}p{#1}}
%%% Double perpendicular symbol (for independence)
\newcommand\independent{\protect\mathpalette{\protect\independenT}{\perp}}
\def\independenT#1#2{\mathrel{\rlap{$#1#2$}\mkern2mu{#1#2}}}
%%% For raw type of inputs (good for esttab command in Stata)
\makeatletter
\newcommand\primitiveinput[1]
{\@@input #1 }
\makeatother
%%% Numbering within sections (Tables)
\numberwithin{table}{section}
%%% Define colors
\definecolor{darkblue}{rgb}{0,0,.4}
\definecolor{winered}{RGB}{198,13,37}
\hypersetup{colorlinks=true, 
			breaklinks=true, 
			citecolor=winered, 
			linkcolor=darkblue, 
			menucolor=darkblue, 
			urlcolor=darkblue}
%%% Return to section updates if available
\newcommand{\returnupdates}{%
Return to \nameref{sec:updates}.%
}
%%% Inserting Stata output (Stars)
\def\sym#1{\ifmmode^{#1}\else\(^{#1}\)\fi}

%-----------------------------------------------------------------
% BEGIN ----------------------------------------------------------
%-----------------------------------------------------------------
%\setlength{\droptitle}{-5em}   										% This is your set screw
\title{La brecha salarial de género: ?`Triple penalización? \\Mujer, madre, e informal}
\author{Jonathan G. Fuentes}
%-----------------------------------------------------------------
\begin{document}
%-----------------------------------------------------------------
\begin{titlepage}
\singlespacing
\maketitle
%-----------------------------------------------------------------
% ABSTRACT
%-----------------------------------------------------------------
\selectlanguage{english} 
\begin{abstract}
	Family formation, child-rearing, and their implications for women's labor market decisions and outcomes have recently been studied as potential `unexplained factors' contributing to the gender pay gap. Opposite to most of the literature available, I study the short-term impacts of motherhood, marriage, and labor market interruptions in Nicaragua, a country with high female labor force participation, shared maternity leave costs (firm and national social security), and a high informality. I provide evidence of unexplained wage differentials by gender and the increase of these differences due to motherhood and employment in the informal sector. I assess a potential mechanism, mothers preferring informal jobs due to the flexibility of working hours. I find that formal jobs represent an absorbing and prefer job status for mothers, even more than for men without children and fathers. Likewise, mothers show higher transition rates from unemployment and inactivity to informal job status (own-account and salaried). These results suggest that mothers are even keener to work, contrary to what an employer (prejudiced or statistically biased) may assume and that the selection into informal or "flexible jobs" is influenced by the lack of better jobs and not by voluntary entry. 
\end{abstract}
\noindent\textbf{JEL Codes:} J13, J16, J46, J69, J71.	\\
\noindent\textbf{Keywords:} gender gap, motherhood penalty, informality, labor transitions \\
 
%-----------------------------------------------------------------
% ABSTRACT
%-----------------------------------------------------------------
\newpage
\selectlanguage{spanish} 
\begin{abstract}
	La formación de familia, la crianza de hijos/as y las implicaciones que tienen para las decisiones de las mujeres y sus resultados en el mercado laboral se han estudiado recientemente como posibles "factores inexplicables" que contribuyen a la brecha salarial de género. Contrario a la mayor parte de la literatura disponible que se ha enfocado en países desarrollados, estudio los impactos de largo plazo que tienen la maternidad, el matrimonio y las interrupciones del mercado laboral en Nicaragua, un país con una alta participación femenina en la fuerza laboral, costos de licencia de maternidad igualmente compartidos (empleador y seguridad social nacional) y una alta informalidad. Proveo evidencia de diferencias salariales entre hombres y mujeres no explicada por características productivas o socioeconómicas, y que dicha brecha aumenta para madres y continua aumentando para madres en el sector informal. Adicionalmente, evalúo un posible mecanismo, las madres prefieren empleos informales debido a la flexibilidad en las horas laborables. En contraste, los resultados sugieren que los empleos formales representan un estado laboral absorbente y preferido por las madres, aún más que para hombres y padres. Así mismo, las madres muestran mayor movilidad laboral desde la inactividad o desempleo hacia trabajos informales (cuenta propia y asalariada). Esto sugiere que las madres están aún más anuentes a trabajar, contrario a lo que un empleador (prejuicioso o estadísticamente sesgado) podría asumir y que la selección en empleos informales o “flexibles” es influenciado por la falta de mejores trabajos y no por la entrada voluntaria de las madres. 
\end{abstract}
\end{titlepage}
%-----------------------------------------------------------------
% INTRODUCTION
%-----------------------------------------------------------------
\newpage 
\doublespacing
\selectlanguage{english}

\section{Introducción}

Ya sea desde un punto de vista normativo o empírico, la discriminación de género es perjudicial para la economía, la sociedad y, aún más importante, para el grupo mayormente afectado, las mujeres. El crecimiento y el desarrollo económico recientes han aumentado el acceso de las mujeres a la educación y la salud; sin embargo, persiste la brecha salarial de género en el mundo desarrollado y en desarrollo. La Organización Internacional del Trabajo (OIT) pronostica que, si se continúa al ritmo actual, se necesitarán hasta 270 años para cerrar dicha brecha \citep{ILO2018}. Sin embargo, mientras la brecha persista, los retornos desiguales de la educación y la productividad siguen representando pérdidas de producción y bienestar para las mujeres y sus hogares.

Recientemente, se han realizado esfuerzos sustanciales e investigaciones innovadoras para comprender las diferencias ``inexplicables'' en los ingresos por género – aquellas diferencias que no pueden ser explicadas por productividad u otras variables observables como lo es el caso de educación o experiencia. Parte de esta investigación se ha concentrado en comprender cómo la disyuntiva entre tiempo e ingresos impuesta por el proceso de crianza de los hijos afecta la brecha salarial de género. Esta literatura muestra que ambos padres enfrentan la decisión de trabajar a tiempo completo u optar por un trabajo flexible y, en muchos casos, salir del mercado laboral. La literatura también muestra que son las madres quien en su mayoría optan por trabajar menos tiempo o hasta salir de la fuerza laboral en comparación con una menor proporción de padres, generando un efecto negativo duradero en los ingresos de las madres durante su estadía en el mercado laboral. \citep{Berniell2021,Boca2013,Goldin2014,Kleven2019}. 

La mayor parte de la investigación empírica en estos temas se ha realizado en países desarrollados, especialmente en Estados Unidos y Europa, donde los estados de bienestar, las políticas de apoyo familiar y las estructuras del mercado laboral difieren sustancialmente de los del mundo en desarrollo. En América Latina, la oferta limitada de "trabajos de calidad" puede jugar un papel importante en la toma de decisiones de los padres para optar por opciones más flexibles durante la crianza de sus hijos. Los trabajos flexibles pueden ser principalmente aquellos trabajos encontrados en el sector informal, lo cual corresponden una gran parte de la oferta laboral de la región, y que en su mayoría implica trabajos sin derecho a seguro médico y/o pensiones, así como mal remunerados. Así mismo, las diferencias en las regulación de licencia de paternidad para madres y padres pueden contribuir al aumento de las brechas salariales de las madres a través de un aumento en discriminación del empleador hacia las mujeres. Por ejemplo, en Nicaragua, mientras que las madres tienen derecho a 12 o hasta 14 semanas de licencia cuando dan a luz, los padres solamente tienen derecho a cinco días según el código laboral vigente.

Nicaragua, con una alta participación laboral femenina (70 por ciento), un alto nivel de informalidad (80 por ciento) y políticas familiares como el costo compartido de la licencia parental entre empleadores (40 por ciento) y el Instituto Nacional de Seguridad Social (60 por ciento), ofrece un entorno apropiado e interesante para investigar la brecha salarial de género, así como las implicaciones de la paternidad y la informalidad. Adicionalmente, el país cuenta con una encuesta cuatrimestral sobre la fuera laboral, la Encuesta de Hogares Continua (ECH), un panel con más de 3,000 hogares, con representatividad a nivel urbano y rural que permite el estudio de dinámicas del mercado laboral, así como controlar por características no observables a nivel de individuo, que son contantes a través del tiempo, y que por ende contribuye a reducir el sesgo de variable omitida. 
 
Al identificar la existencia de una brecha salarial de género y sus cambios en diferentes grupos en el mercado laboral, este estudio tiene como objetivo proveer evidencia sobre la posible triple penalidad que las mujeres podrían enfrentar en el mercado laboral de Nicaragua debido a la maternidad y su inserción en el sector informal. Por lo tanto, en el presente articulo analizo la brecha salarial de género, primero a través de la estimación de ecuaciones de salariales à la Mincer controlando por variables a nivel individual, de hogar, del lado de la oferta y por characteristicas no observables a nivel de individuo, y segundo a través la estimación de descomposiciones Blinder-Oaxaca para diferentes submuestras (mujeres y hombres, madres y padres, y madres y padres en el sector informal) con el fin de identificar brechas salariales después de resolver por la potencial amenaza de autoselección de la muestra. 

Al estimar los modelos antes explicados encuentro evidencia de la existencia de una brecha de género entre hombres y mujeres cercana al 19\%, y que ésta brecha aumenta al comparar madres y padres en el sector formal. La brecha entre padres y madres en el sector informal parece ser menor que en el sector formal, sin embargo, este resultado es causado por la diferencias en atributos y cualidades como el nivel de educación, la cual desaparece después de controlar por selección de la muestra. Por lo tanto, la evidencia provista en el presente estudio favorece a la hipotesis de una triple penalidad. Las mujeres en Nicaragua enfrentan una penalidad adicional por su condición de madre así como por su condición de informalidad.  

Adicionalmente, investigo la posible naturaleza de dicha brecha de género al proveer evidencia de una de las potenciales explicaciones, la creencia de que las mujeres, y especialmente las madres, prefieren trabajos flexibles antes de trabajos tiempo completo para así poder dedicar tiempo a las tareas reproductivas del hogar. Analizo patrones de mobilidad laboral para hombres, mujeres, padres y madres, enfocandome en identificar diferencias en sus permanencia en empleos formales o, equivalentemente, transiciones hacia empleos flexibles, en este caso representados como empleos informales. Al usar matrices de transiciones condicionales encuentro que las tasas de permanencia en empleos formales es mayor para madres comparado con hombres y padres. Así mismo, las transiciones fuera de inactividad y desempleo hacia la actividad en la fuerza laboral son mayores para madres en comparación con mujeres sin hijos, hombres y padres. Las mujeres incrementan sus probabilidad de entrar en trabajos informales (cuenta propia o asalariada) cuando se convierten en madre. 

El resto del artículo está estructurado de la siguiente manera. La sección dos ofrece una revisión de la literatura existente sobre la brecha salarial de género, su medición y los factores actuales que se cree que la explican, con un enfoque principal en la investigación que examina los factores "inexplicables". En la sección tres, explico los datos y los modelos econométricos utilizados en el estudio. En la sección cuatro, proporciono una breve descripción del mercado laboral nicaragüense con un enfoque en las mujeres. La sección cinco se presenta los principales resultados de las estimaciones y la última sección seis concluye y ofrece recomendaciones de política económica y social.

%-----------------------------------------------------------------
% LITERATURE REVIEW
%-----------------------------------------------------------------
\section{Literature review}

La brecha de género se refiere a las diferencias sistemáticas que hombres y mujeres, idénticos en cuanto a su capacidad productiva, enfrentan en el mercado laboral debido a características no productivas. Estas diferencias aparecen en los porcentajes de hombres y mujeres en la fuerza laboral, los tipos de ocupaciones que eligen y la diferencia en las remuneraciones de hombres y mujeres \citep{Goldin2008}.

El diferencial salarial total entre hombres y mujeres se puede descomponer en una parte explicada por las diferencias en las características productivas y un componente residual inexplicable. El papel de las diferencias de género en este residuo inexplicable es a menudo se denomina efecto de discriminación. La investigación empírica sobre las brechas salariales de género se ha centrado tradicionalmente en el rol de los factores específicos de género, en particular las diferencias en las calificaciones y las diferencias en el trato a trabajadores masculinos y femeninos igualmente calificados (es decir, discriminación en el mercado laboral) \citep{Blau1992,Weichselbaumer2005}

El análisis económico de la discriminación se remonta al trabajo inicial sobre la economía de la discriminación de \citet{Becker1985} y la vasta literatura que siguió. La teoría de Becker sobre la discriminación en el mercado laboral se basa en el concepto de discriminación basada en el gusto. Aplicado a las diferencias de género, este concepto se refiere a la noción de prejuicio de género y resultados económicos. Becker monetiza este prejuicio y determina sus diferentes fuentes, clasificadas como discriminación del empleador, discriminación del empleado y discriminación del cliente \citep{Borjas2016}.

Además, o incluso en ausencia de discriminación basada en prejuicios, las diferencias en las habilidades o la productividad son atribuidas a diferentes a géneros por parte del empleador. Los empleadores tienen información imperfecta sobre la productividad individual y solo conocen la productividad promedio a nivel de grupo, por ejemplo, las mujeres. Así, la productividad asumida para un trabajador por parte del empleado da lugar a diferencias inexplicables. La noción de esta teoría se basa en el trabajo inicial de \citet{Arrow1973} y \citet{Phelps1972} citado por \citet{Altonji1999} y es conocida como discriminación estadística.

La brecha salarial de género, y los factores que la determinan, han estado sujetos a cambios constantes a lo largo del tiempo, a un ritmo similar al que cambia el mercado laboral. Hasta la fecha, los factores tradicionales como la afiliación sindical, la raza, el estatus migratorio o la religión, y especialmente las variables relacionadas con el capital humano, es decir, el nivel educativo y la experiencia, explican muy poco de los diferenciales de ingreso entre hombres y mujeres \citep{Blau2017,Weichselbaumer2005}. Recientemente, la literatura ha analizado cada vez más otros factores debido a nuevos fenómenos observados en el mercado laboral o a las mejoras en los métodos utilizados para estudiar las contribuciones de estos factores a las diferencias salariales.

A lo largo de las mejoras del siglo pasado, la brecha de género en el acceso a la educación casi ha desaparecido y la del nivel de educación incluso se ha revertido en países desarrollados y en desarrollo \citep{Becker2010,Gaddis2013,Goldin2006}.  Esta tendencia también se observa en el caso del nivel de experiencia en el mercado laboral. En los Estados Unidos, en 2011, las mujeres tenían un tres por ciento más de probabilidades de obtener un título avanzado que los hombres, y los años de experiencia a tiempo completo de los hombres (17.8 años) eran solo 1.4 años más altos que los de las mujeres \citep{Blau2017}. Así, en lugar de explicar las diferencias salariales, el nivel de educación y la experiencia actualmente representan más bien un factor reductor de dicha brecha.

Hasta el presente, el foco de atención de la evidencia empírica en género y el mercado laboral se ha desplazado hacia el análisis de tendencias y fenómenos recientes, como el estancamiento de la participación laboral femenina o el aumento de la liberalización comercial y su impacto en las mujeres. Asimismo, se han realizado importantes esfuerzos e innovaciones con el objetivo de esclarecer la parte inexplicable de la brecha salarial entre hombres y mujeres y los factores que la componen, incluyendo la formación de la familia y los factores de procreación y crianza.

%-----------------------------------------------------------------
\subsection{Interrupciones del mercado laboral}
%-----------------------------------------------------------------
Las diferencias en la permanencia en el mercado laboral han estado en el centro del debate como posibles factores determinantes de la brecha salarial de género \citep{Mincer1974}. Las interrupciones en el mercado laboral de las mujeres, principalmente debido a la formación de una familia, es decir, el matrimonio y/o el parto, todavía contribuyen significativamente a las diferencias salariales entre hombres y mujeres \citep{WorldBank2012}. Por ejemplo, \citet{Goldin2014} encontró que las interrupciones laborales para los profesionales de MBA y finanzas en los Estados Unidos contribuyen hasta un 30 por ciento de la brecha de género en los ingresos.

Aunque la familia, su estructura y funcionamiento han cambiado con el tiempo, aún se observan diferencias entre países y regiones derivadas principalmente de las instituciones formales e informales que prevalecen en sus países. En Argentina, Brasil, Ghana, México, Serbia y Tailandia, las diferencias entre hombres y mujeres en el uso del tiempo para el trabajo reproductivo afectan negativamente su probabilidad de transitar hacia "buenos trabajos" y aumentan sus probabilidades de estar en el sector informal de trabajadores autónomos o la fuerza laboral inactiva \citep{Bosch2010}.

La magnitud de la contribución de las interrupciones laborales depende de su duración y como son cuantificadas. La evidencia empírica soporta la hipótesis de heterogeneidad en el impacto de interrupciones en los salarios de las mujeres dependiendo de su duración. Interrupciones cortas están generalmente asociadas con impactos positivos, mientras que interrupciones largas con efectos negativos \citep{Blau2000,Ruhm1998,Waldfogel1998}. No obstante, la mayor parte de la literatura empírica se encuentra con el problema de la carencia de una forma adecuada de medir las interrupciones laborales y su duración, lo que conduce a utilizar sustitutos imperfectos. \citet{Nordman2016} explotan una nueva base de datos de Madagascar que les permitió comparar la contribución de la experiencia real y potencial (proxy) a la brecha salariar de género proporcionando evidencia de que el uso de proxis como experiencia laboral subestima la contribución de las interrupciones del trabajo a las diferencias salariales de género.

La experiencia adquirida en el trabajo tiene una contribución significativa a la brecha de ingresos entre hombres y mujeres a lo largo de sus carreras profesionales \citep{Kleven2019}.  Dichas diferencias presentan magnitudes heterogéneas entre regiones, representando desde un 1 por ciento de los determinantes del salario para las mujeres en Europa y hasta un 10 por ciento en los países en desarrollo \citep{Weichselbaumer2005}. 
%-----------------------------------------------------------------
\subsection{Penalidad por maternidad}                        
%-----------------------------------------------------------------
La evidencia empírica sugiere que, además del impacto de la permanencia en el mercado laboral, los roles de género y las decisiones sobre la formación de la familia podrían afectar los salarios relativos de las mujeres \citep{Blau2017}. Este fenómeno se denominó inicialmente brecha familiar, y representa la diferencia salarial entre mujeres con hijos y mujeres sin hijos \citep{Waldfogel1998}. Actualmente, es más conocido en la literatura como la penalidad de maternidad.

\citet{Budig2001} utilizan un modelo de efectos fijos para Estados Unidos con la finalidad de medir la magnitud de la penalidad de maternidad y obtuvieron resultados que muestran una penalización salarial del 7 por ciento por hijo. También encontraron que la penalidad es mayor para las mujeres casadas que para las solteras y que las mujeres con (más) hijos tienen menos años de experiencia laboral. Así mismo, observaron que, después de controlar por experiencia, se mantiene la penalidad a un nivel cercano al 5 por ciento por hijo.

Utilizando datos de siete países con diferentes estados de bienestar, \citet{SigleRushton2007} encontraron que las brechas en el ingreso familiar son menores en los países nórdicos, comparado a los angloamericanos y ambas menores que la observadas en los países de Europa continental, concluyendo, por lo tanto, que el efecto de la maternidad difiere según el estado de bienestar.

\citet{Kleven2019} utilizan datos daneses y encuentran que la llegada de un niño crea una brecha de género en los ingresos de alrededor del 20 por ciento, causada principalmente por interrupciones del trabajo, horas de trabajo y niveles de remuneración, y que estas afecta a las madres, pero no a los padres. Las brechas de género una vez ocurren son muy estables en el tiempo y las mujeres no muestran signos de recuperación en el mercado laboral incluso diez años después del primer hijo. \citet{Kleven2019} también encontró que la penalidad por cada hijo se transmite de generación en generación, a las hijas y no a los hijos, a través de la influencia que el entorno tiene en la formación de preferencias de las niñas sobre la familia y la carrera.

Estimar y establecer un efecto causal de la pena de maternidad sigue siendo controversial debido a factores como la autoselección de las mujeres, es decir, las mujeres con salarios más bajos tienen menores costos de hijos y viceversa. Varios factores surgen de la literatura como una posible explicación de la pena de maternidad, entre ellos, la discriminación de los empleadores y la disyuntiva entre horas de trabajo y salarios más altos o trabajos con horario favorables para las madres \citep{Budig2001}.

La evidencia experimental en el ámbito de discriminación ha tenido como objetivo evaluar la existencia de tal y sus posibles determinantes. \citet{Correll2007} llevaron a cabo un experimento de laboratorio en el que se evaluaron los currículums de solicitantes de empleo del mismo sexo igualmente calificados, que solo se diferenciaban por su estado parental. Los autores observaron que la recomendación salarial fue menor para madres, mientras que los padres no fueron penalizados. Además, \citet{Benard2010} examinaron si las madres enfrentan discriminación al momento de ser evaluadas en el mercado laboral, aún al brindar evidencia indiscutible de que son competentes y están comprometidas con el trabajo remunerado. Sus resultados sugieren que las mujeres evaluadoras aplican "discriminación normativa", encontrando a las madres menos cálidas, menos agradables y más hostiles interpersonalmente que las trabajadoras similares que no son madres. Al contrario, \citet{Williams2015} no encontraron evidencia de discriminación en el mercado académico y concluyeron que la discriminación hacia las madres podría resultar de discriminación estadística debido a la percepción de los empleadores de las diferencias de productividad entre madres y no madres.

Las diferencias de género en el uso del tiempo, particularmente en actividades no remuneradas, impactan en los resultados que las mujeres obtienen en el mercado laboral de varias maneras \citep{Becker1985,Blau2017}. Por ejemplo, las largas horas que las mujeres casadas o las madres dedican a actividades reproductivas podrían reducir el esfuerzo que dedican a sus trabajos en el mercado. En esa misma línea, \citet{Hersch2002} encuentran que las tareas domésticas, especialmente las tareas de rutina diaria como cocinar y limpiar, perjudican los salarios independientemente del estado civil en los Estados Unidos. Además, los autores proporcionan evidencia de que el control del tiempo de trabajo doméstico aumenta el componente explicado de la brecha salarial de género en 14 puntos porcentuales. \citet{Cha2014} examinan el papel de un aumento en la prevalencia de largas horas de trabajo y la brecha salarial de género durante el período 1979-2009, encontrando que la diferencia en las horas trabajadas aumentó la brecha salarial de género en aproximadamente un 10 por ciento del cambio total durante este período.
%-----------------------------------------------------------------
\subsection{Informalidad}
%-----------------------------------------------------------------
Los mercados laborales de los países en desarrollo se caracterizan por altos niveles de informalidad. Por ejemplo, en América Latina, el llamado sector informal comprende una gran proporción (30 a 70 por ciento) de la fuerza laboral ocupada \citep{Maloney2004}. Los trabajos en el sector informal difieren de los trabajos formales en dimensiones como una protección social, menores ingresos laborales, menores perspectivas de carrera, más flexibilidad y jornadas laborales más cortas \citep{Berniell2021}. Por ende, las principales conclusiones de la vasta literatura sobre la brecha salarial de género que se han centrado principalmente en Estados Unidos y Europa podrían tener diferentes aplicaciones en países con altas tasas de informalidad.

\citet{Tansel2001} utilizó la descomposición de Oaxaca-Blinder para estimar las diferencias salariales entre hombres y mujeres según el diferente estado de cobertura del seguro social en Turquía. En este contexto, el autor concluyó que, en el sector con cobertura de seguridad social, los salarios de los hombres son aproximadamente dos veces más altos que los de las mujeres. Para los trabajadores asalariados fuera de la seguridad social, los salarios de los hombres están casi a la par con los de las mujeres. Estos resultados sugieren una segmentación para los hombres en el sector formal e informal y una discriminación sustancial para las mujeres en el sector privado con cobertura de seguridad social.

De manera similar, \citet{Ruzik2010} analizaron la brecha salarial de género en el sector informal de Polonia utilizando regresiones por cuantiles, en este caso analizando trabajadores formalmente registrados y no registrados, encontrando que la desigualdad de ingresos entre mujeres y hombres no registrados es más pronunciada en la cola inferior de la distribución de ganancias. En el caso de los empleados formales, la desigualdad en la parte superior de la distribución tiende a ser mayor, lo que confirma la existencia de un 'techo de cristal'. El estudio también propone que una posible explicación de los resultados es la falta de regulaciones de salario mínimo en el mercado informal y la mayor flexibilidad al momento de negociar y decidir sobre salarios en los cuantiles superiores.

Por el contrario, \citet{Yahmed2018} estudió cómo la desigualdad de género difiere entre trabajadores formales e informales en Brasil, y encontró que la brecha salarial de género bruta es aproximadamente la misma en trabajos informales y trabajos formales. Sin embargo, también encuentra que este resultado es provocado por la diferencia en los procesos de selección masculina y femenina en cada uno de los sectores. Por lo tanto, después de controlar las características observables, la brecha salarial de género ajustada es, en promedio, alrededor del 24 por ciento entre los empleados formales y alrededor del 20 por ciento entre los empleados informales.

Por último, \citet{Berniell2021} investigan más a fondo el efecto de las diferencias de género por estado de formalización y el efecto de la maternidad para Chile y otros países en desarrollo de la Organización para la Cooperación y el Desarrollo Económico (OCDE), encontrando que la maternidad produce una reducción considerable en los ingresos laborales de las madres chilenas y que esta reducción es asociada a una menor participación en la fuerza laboral y una caída en el empleo formal. Asimismo, el estudio encuentra que la penalidad por maternidad en Chile es menor que en Estados Unidos, pero es mayor que en Dinamarca.
%-----------------------------------------------------------------
\subsection{Medición}
%-----------------------------------------------------------------
Diferentes métodos han sido desarrollados y utilizados para analizar la discriminación empíricamente. La forma más común de analizar la discriminación basada en el género es comparar los ingresos masculinos y femeninos manteniendo la productividad constante. Otra manera de hacerlo sería incluir una variable dummy representando el género del individuo en un modelo de regresión salarial únicamente. Sin embargo, el procedimiento estándar para investigar las diferencias en los salarios es mediante el uso de métodos de descomposición \citep{Fortin2011,Weichselbaumer2005}.

Varios métodos de descomposición se han desarrollado desde el trabajo seminal de \citet{Oaxaca1973} y \citet{Blinder1973}. Sin embargo, la denominada descomposición Blinder-Oaxaca sigue siendo el enfoque más utilizado a la fecha. Este procedimiento permite descomponer el diferencial salarial entre hombres y mujeres en una parte explicada debido a diferencias en las características y un residuo inexplicable \citep{Fortin2011,Weichselbaumer2005}.
%-----------------------------------------------------------------
\subsection{La Brecha Salarial De Género En Nicaragua}
%-----------------------------------------------------------------
Pocos estudios han analizado la brecha salarial de género en Nicaragua. \citet{Weichselbaumer2005} realizan un meta-análisis y concluyen que una parte sustancial de la brecha salarial total puede atribuirse a diferencias en el capital humano, debido sus hallazgos que indican que las mujeres tienen diferencias en características socioeconómicas que se resultan en una menor productividad comparado con los hombres. Asimismo, \citet{Monroy2008} estima las diferencias brutas en los ingresos de los trabajadores y trabajadoras por ocupación, encontrando una diferencia promedio del 20 por ciento. Adicionalment, después de estimar las diferencias brutas en los ingresos por estatus de formalización, encuentra que la brecha en el sector informal (18 por ciento) es mayor que en el sector formal (7 por ciento).

De manera similar, el informe del \citet{PNUD2014} sobre el mercado laboral nicaragüense desde una perspectiva de género utiliza descomposiciones de Blinder-Oaxaca y encuentra que el ingreso mensual de los hombres es mayor que el de las mujeres en más del 30 por ciento tanto para las áreas rurales como urbanas. No obstante, la magnitud de coeficientes encontrados podría atribuirse a problemas de autoselección.

El presente artículo contribuye a la literatura sobre la brecha salarial de género en Nicaragua mediante la implementación de técnicas econométricas mejoradas y al aplicarlo en un conjunto de datos de panel para estimar las diferencias salariales mientras se controlan las características individuales observables y no observables, variables a nivel de hogar y posibles problemas de autoselección. Además, la característica del panel de la encuesta permite el estudio de las interrupciones laborales en los salarios y el análisis de los potenciales canales de la penalización por maternidad, en particular la probable selección de madres en trabajos más flexibles o informales. Por último, el analizar la brecha salarial de género después de resolver la autoselección podría proveer evidencia sobre el efecto particular de la licencia de maternidad con costos compartidos al comparar los sectores formales e informales. Esta política tiene el potencial de inducir prejuicios y discriminación por parte de los empleadores, y por ende el estudio de sus implicaciones es del interés de los hacedores de política. 

%-----------------------------------------------------------------
% DATA AND METHODOLOGY
%-----------------------------------------------------------------
%-----------------------------------------------------------------
\section{Datos y metodología}
%-----------------------------------------------------------------
Utilizo los datos longitudinales de la Encuesta Continua de Hogares (ECH) para estimar los modelos descritos en este documento. La ECH recopila trimestralmente información sobre las condiciones socioeconómicas y del mercado laboral con un panel rotativo en más de siete mil hogares enfocados en la fuerza laboral activa. Es representativo del nivel rural y urbano y proporciona suficiente información sobre trabajadores formales e informales.

La fuente principal de información de las siguientes estimaciones son las bases de datos de los cuatro trimestres de 2012. La unidad de análisis aplicada son los individuos pertenecientes a la fuerza laboral, que contienen personas entre 14 y 65 años (80,863 observaciones). El intervalo de edad ha sido definido de acuerdo a la edad mínima utilizada a nivel nacional para definir la fuerza laboral por el Banco Central de Nicaragua y el INIDE \citep{BCN2009}.

La principal variable de análisis es salarios, la cual se refiere a todos los ingresos laborales de las ocupaciones primarias y secundarias, y se ha construido a partir de la información recopilada por la encuesta. La encuesta no incluye información directa sobre las madres. Por lo tanto, codifico como madre a una mujer que es pareja del jefe de familia o que es la jefa de familia cuando hay hijos en el hogar. Los padres se codifican de manera similar. Las interrupciones de trabajo se codifican utilizando las características del panel. La variable dummy para interrupciones laborales toma un valor de uno cuando un individuo en el trimestre anterior de la encuesta estaba desempleado, fuera de la fuerza laboral o inactivo y en la encuesta actual ha cambiado a otro tipo de situación laboral en la actividad.

El estado laboral ha sido analizado de acuerdo con la XIX Conferencia Internacional de Estadísticos del Trabajo (CIET) \citep{ILO2013}, ya que las características de la encuesta ECH lo permiten. La informalidad y la clasificación de cada uno de los estados laborales se han medido según la 17ª CIET \citep{ILO2013}. La clasificación de la industria se ha elaborado utilizando las Naciones Unidas (\citeyear{Nations2008}) y la Organización para la Cooperación y el Desarrollo Económicos \citep{OECD1997}.

Las estimaciones primarias aplicadas en este estudio tienen como objetivo proporcionar evidencia sobre la existencia de una brecha salarial de género en Nicaragua, así como resaltar sus magnitudes y posibles factores asociados con la misma. Dichas estimaciones son: i) determinantes de los salarios mensuales y ii) una descomposición Blinder-Oaxaca. Ambas estimaciones también analizan las diferencias de género para diferentes estatus de formalización. Además, para estudiar una de las potenciales causas de la penalidad de maternidad, mujeres optando por "trabajos flexibles", iii) estimo matrices de transición que siguen un proceso de transición de Markov para proporcionar evidencia sobre la movilidad laboral de mujeres y madres.

Se implementan ecuaciones salariales à la Mincer para los ingresos mensuales. La primera estimación es un MCO utilizando una muestra que incluye a hombres y mujeres, con la intención de proporcionar evidencia inicial del impacto del género en la determinación de los salarios, seguida de modelos separados para hombres y mujeres. Las ecuaciones salariales incluyen variables individuales, del hogar y del lado de la oferta y se expresan de la siguiente manera:
%
\begin{equation}
    LnWages_{it} = \beta_{0} + \beta_{1} X'_{it} + \beta_{2} X'_{ht} + \beta_{3} X'_{st} + \varepsilon_{it}, 
\end{equation}
%
El conjunto principal de variables de control incluye variables a nivel individual y del hogar $\beta X'_{it}$, específicamente la situación laboral de acuerdo con la clasificación de la OIT, edad, edad al cuadrado, área de residencia, un dummy de casado, nivel educativo, industria, un dummy de madre, un dummy de padre, un dummy que representa la interrupción del trabajo, indicando aquellos que estuvieron el trimestre anterior fuera de la fuerza laboral, inactivos o desempleados, y un dummy para quienes están actualmente estudiando o en capacitación. A nivel de hogar $\beta X'_{ht}$, incluyo los siguientes controles: número de hijos y variables dummy para un hogar que recibe remesas, pensión y beneficiario de programas de escuelas públicas y tamaño del hogar. Asimismo, incluyo variables dummy (efectos fijos) para los estratos económicos identificados en el marco muestral y para cada uno de los trimestres de la encuesta para controlar por características socioeconómicas o posibles choques idiosincrásicos en el área de residencia o en el tiempo (trimestres).

Posteriormente, estimo las mismas ecuaciones salariales utilizando efectos fijos individuales con el objetivo de controlar por el sesgo inducido por variables no observadas como lo es la diferencia en habilidades de los trabajadores que podría sesgar las estimaciones de MCO \citep{Angrist2008}. Las ecuaciones salariales con efectos fijos están estimadas usando las siguiente especificación:
%
\begin{equation}
    LnWages_{it} = \alpha_{i} + \gamma_{t} + \beta_{0} X_{it} + \beta_{1} X_{ht} + \beta_{2} X_{st} + \varepsilon_{it}, 
\end{equation}
%
\noindent $\alpha_{i} = \alpha + A'_{i}\delta$, representa efectos individuales no observados, es decir, coeficientes sobre variables dummy para cada individuo \citep{Angrist2008}.

Las estimaciones de las brechas salariales de género se realizan siguiendo el trabajo de \citet{Blinder1973} y \citet{Oaxaca1973} quienes propusieron un procedimiento de descomposición que divide el diferencial salarial entre dos grupos, una parte que se "explica" por las diferencias grupales en las características de productividad, tales como como educación, y una parte residual que no puede explicarse por tales diferencias en los determinantes salariales. Esta parte "inexplicable" se utiliza a menudo como una medida de discriminación, pero también incluye los efectos de las diferencias de grupo en predictores no observados \citep{Jann2008}.

Con base en las ecuaciones salariales iniciales, la descomposición de Oaxaca-Blinder se puede expresar de la siguiente manera:
%
\begin{equation}
    \overline{W}_{m} - \overline{W}_{f} = (\overline{X}_{m} - \overline{W}_{f})\hat{\beta}_{m} + (\hat{\beta}_{m} - \hat{\beta}_{f})\overline{X}_{f} = E + U 
\end{equation}
%
\noindent Donde $\overline{W}_{m} - \overline{W}_{f}$ denota el logaritmo medio de los salarios y las características de control para cada grupo. Esta diferencia equivale a dos términos, un primer término que representa el efecto de diferentes características productivas $(\overline{X}_{m} - \overline{W}_{f})\hat{\beta}_{m}$, y el segundo término representa el residual no explicado $(\hat{\beta}_{m} - \hat{\beta}_{f})\overline{X}_{f}$, las diferencias en los coeficientes estimados para ambos grupos se denomina generalmente efecto de discriminación \citep{Weichselbaumer2005}.

La descomposición inicial del salario se presenta para una muestra combinada de individuos que analizan a los hombres y mujeres con características similares, seguida de un análisis de descomposición por paternidad y estatus de formalización. Estos modelos pretenden proporcionar estimaciones brutas del diferencial salarial y sus componentes. No obstante, los salarios solo se observan para las personas que participan en la fuerza laboral, lo que genera posibles problemas de autoselección. Por lo tanto, para controlar este problema, utilizo los procedimientos de corrección de selección de muestras propuestos por \citep{Heckman1979}. Este procedimiento implicó la modelización de la participación laboral en función de la edad, el estado civil y el número de hijos.

Para investigar la probabilidad de transitar entre diferentes estatus laborales, el estatus flexible y no flexible (aproximados como empleos formales e informales), se estimaron matrices de transición intertrimestrales que brindan una impresión general de los patrones de movilidad laboral por género y estatus parental. Según \citet{Nichols2014}, una matriz de transición tiene como objetivo medir un estado en el tiempo $t-1$ y nuevamente en el tiempo $t$. La matriz de transición es un proceso de Markov que estima la probabilidad condicional de encontrar un trabajador en un estado $j$ en el tiempo $t$, condicionado al hecho de que el individuo estaba en el estado $i$ en el tiempo $t-1$. La suma de los elementos en cada fila de la matriz de transición es igual a 1, mientras que la diagonal representa la permanencia en el estado laboral original, lo que ayuda a ilustrar aquellos estados laborales con mayor persistencia.

Las matrices de transición se determinan de la siguiente manera:
%
\begin{equation}
    p_{ij}(t, t + 1) = Pr(X_{t + 1} = j | X_{t} = i)   
\end{equation}
%
Las matrices de transición se estiman con un panel desbalanceado de la ECH con datos de los cuatro trimestres de 2012, para lo cual se aplicaron métodos para rectangularizar los datos en el software estadístico con el fin de obtener matrices de transiciones de Markov. Para estas matrices, se consideraron seis clasificaciones de situación laboral: trabajadores formales asalariados, trabajadores informales, autónomos / trabajadores por cuenta propia, autónomos informales / trabajadores por cuenta propia informal, desempleados y fuera de la fuerza laboral o población inactiva.\footnote{Fuera de la fuerza laboral y población inactiva fueron combinados debido al pequeño tamaño de la muestra de población inactiva.}
%-----------------------------------------------------------------
% MERCADO DE TRABAJO EN NICARAGUA
%-----------------------------------------------------------------
%-----------------------------------------------------------------
\section{Mercado de trabajo en nicaragua -- Un Enfoque de Género}
%-----------------------------------------------------------------
En esta sección se brinda una breve descripción del mercado laboral nicaragüense con base en la Encuesta Continua de Hogares (ECH) del último trimestre de 2012. Todas las estadísticas descriptivas utilizan factores de expansión para facilitar la inferencia sobre la población. Para obtener una descripción más detallada, consulte la tabla 1 del apéndice.

La muestra total de la encuesta se divide en 49 por ciento de hombres y 51 por ciento de mujeres. Entre los hombres de la encuesta, el 63 por ciento pertenece a la población en edad de trabajar y entre las mujeres, esta proporción representa el 65 por ciento. La población ocupada, siguiendo la definición de la 19 CIET, asciende al 90 por ciento de los hombres y al 71 por ciento de las mujeres en la población en edad de trabajar. A pesar de las diferencias en la participación, las tasas de informalidad entre ambos grupos son similares, 80 por ciento para hombres y 81 por ciento para mujeres. De manera similar, la proporción de la población que vive en el sector rural representa el 44 por ciento de los hombres y el 41 por ciento de las mujeres.

En promedio, el ingreso mensual observado para los hombres es de 3,346 córdobas, lo que representa alrededor de US\$142 utilizando tipo de cambio al momento de la encuesta. Para las mujeres, el ingreso promedio es de 2,824 en moneda local, lo que corresponde a US\$120. Por tanto, los ingresos de las mujeres representan el 84 por ciento de los ingresos de los hombres. Sin embargo, el promedio de horas que trabaja una mujer (30 horas) representa el 75 por ciento del promedio de horas trabajadas por un hombre (41 horas).

Hombres y mujeres presentan niveles similares de escolaridad en cuanto a la finalización de los grados de primaria, secundaria y superior. La proporción de hombres sin educación formal es del 24 por ciento en comparación con el 23 por ciento de las mujeres. Asimismo, el 11 por ciento de los hombres completó la escuela primaria en comparación con el 10 por ciento de las mujeres. Por el contrario, el ocho por ciento de las mujeres completó la escuela secundaria en comparación con el siete por ciento de los hombres y el 10 por ciento de las mujeres logró un título avanzado en comparación con el nueve por ciento de los hombres. Por tanto, la diferencia parece no ser significativa; esto podría indicar que cualquier brecha salarial no se atribuye a diferencias educativas.

En términos de segregación ocupacional, las principales diferencias se observan en la industria agrícola y pesquera, donde el 45 por ciento de los hombres están empleados en comparación con solo el 21 por ciento de las mujeres. De manera similar, el sector de la construcción emplea al siete por ciento de la fuerza laboral masculina y solo al uno por ciento de la fuerza laboral femenina. En contraste, la industria del comercio, hoteles y restaurantes emplea al 38 por ciento de la fuerza laboral femenina y sólo al 19 por ciento de la fuerza laboral masculina. Además, la industria de la educación, la salud y la protección social contrata al 15\% de la fuerza laboral femenina y al 6\% de la masculina.

Además, los datos de la encuesta muestran a nivel del sector macroeconómico que aproximadamente el 66 por ciento de las mujeres se insertan en el sector terciario, especialmente en el comercio, ventas y servicios personales, y generalmente como trabajadoras por cuenta propia, mientras que este sector solo emplea al 38 por ciento. de la fuerza laboral masculina. Por el contrario, el sector primario contiene el 45 por ciento de la fuerza laboral masculina y el 21 por ciento de la fuerza laboral femenina. El sector secundario muestra una participación más equilibrada, empleando al 17 por ciento de los hombres y al 13 por ciento de las mujeres en la fuerza laboral.

%-----------------------------------------------------------------
% RESULTS: EVIDENCE OF A GENDER WAGE GAP
%-----------------------------------------------------------------
%-----------------------------------------------------------------
\section{Results -- Evidence of a gender wage gap}
%-----------------------------------------------------------------
\subsection{Determinantes de los salarios}
%-----------------------------------------------------------------
Para inspeccionar el impacto del género en la determinación del salario y la posible discriminación basada en el género, estimo ecuaciones salariales inicialmente utilizando la muestra combinada de hombres y mujeres, y luego, realizando una estimación de los determinantes salariales por género mientras se controla por sesgo de variable omitida con un modelo con efectos fijos. Comparé el modelo de efectos fijos con un efecto aleatorio utilizando la prueba de Hausman, lo que resultó en que el modelo de efectos fijos es el modelo preferido.

Como resultado del modelo inicial de MCO, la variable dummy femenina tiene un efecto significativo y negativo sobre el logaritmo del ingreso laboral neto promedio. Por lo tanto, ser mujer reduce el ingreso laboral neto promedio en aproximadamente un 20 por ciento. Además, estar casado tiene un efecto positivo y significativo en los ingresos de alrededor del 8 por ciento. La dummy para madre es insignificante, mientras que ser padre tiene un efecto significativo y positivo de alrededor del 8 por ciento en los ingresos laborales. Además, haber estado desempleado, fuera de la fuerza laboral o inactivo en el trimestre anterior tiene un efecto significativo y negativo en los ingresos, reduciéndolos en aproximadamente un 30 por ciento.

Otras variables de control como la edad, la edad al cuadrado, la situación laboral, la asistencia a la escuela y las variables dummy que muestran que el hogar recibe ingresos no laborales como los programas sociales y de pensiones, son significativas y muestran direcciones y magnitudes según lo predicho por la literatura. En el cuadro 2b del apéndice se pueden encontrar más detalles de las ecuaciones salariales, incluidos los coeficientes para todas las variables de control.
%
%-----------------------------------------------------------------
% Tab 1
%-----------------------------------------------------------------
\begin{table}[H]
    \singlespacing
	\small
	\centering 
	\begin{adjustbox}{max width=\textwidth}
		\begin{threeparttable}
			\caption{Ecuaciones de Mincer}
			\label{tab:mincer}
			\begin{tabular}{@{}l*{6}{c}@{}}
				\toprule
				 		        					& 
				\multicolumn{3}{c}{MCO Agruado} 	&
				\multicolumn{3}{c}{Efectos Fijos}	\\ \cmidrule(lr){2-4} \cmidrule(l){5-7}
                     	 			&
                Muestra total    	&
                Mujeres 			&
                Hombres 			&
                Muestra total 		&
                Mujeres       		&
                Hombres         	\\
				Variables			&
				(1) 				& 
				(2) 				& 
				(3) 				& 
				(4) 				& 
				(5) 				& 
				(6) 				\\
				\midrule
                \primitiveinput{tables/mincer.tex}
                                        	&    &    &    &    &    &        \\
                 Controles Individuales    	& Si & Si & Si & Si & Si & Si     \\
                 Controls Hogar     		& Si & Si & Si & Si & Si & Si     \\
                 Controles oferta   		& Si & Si & Si & Si & Si & Si     \\			
				\bottomrule
			\end{tabular}
			\begin{tablenotes}
				\setlength\labelsep{0pt}
				\footnotesize
				\item \textit{Notas}: 
			\end{tablenotes}
		\end{threeparttable}
	\end{adjustbox}
\end{table}
%
En las regresiones de MCO por submuestra por género, se encuentra que la variable dummy de casado solo es significativa para los hombres, lo que aumenta sus ingresos laborales, pero no se observa lo mismo para el caso de las mujeres. Además, la variable dummy de padre siguió siendo significativa y se asoció positivamente con los ingresos, mientras que la dummy de la madre se volvió insignificante.

No obstante, el modelo de efectos fijos muestra que los cambios en la maternidad, es decir, convertirse en madre, tiene un efecto significativo y negativo en los ingresos, lo que representa una reducción de alrededor del 15 por ciento. Por el contrario, los cambios en la variable dummy del padre no son significativos. Además, la interrupción en la permanencia en la fuerza laboral tiene un efecto significativo y negativo solo para las mujeres, lo que implica, en promedio, una reducción del 26 por ciento de los ingresos de las mujeres cuando están fuera de la fuerza laboral, desempleadas o inactivas en el trimestre anterior.
%-----------------------------------------------------------------
\subsection{Descomposiciones Blinder-Oaxaca}
%-----------------------------------------------------------------
Los métodos de descomposición son útiles para investigar más a fondo las diferencias en los salarios. El método Blinder-Oaxaca permite descomponer las diferencias salariales entre hombres y mujeres en una parte explicada por diferencias de características y una parte residual no explicada. El efecto de diferentes características productivas se denomina efecto dotación, y el residuo inexplicable, las diferencias en los coeficientes estimados para ambos grupos, a menudo se denomina "efecto de discriminación" \citep{Weichselbaumer2005}.

En la estimación de la descomposición Blinder-Oaxaca, no puedo controlar las posibles violaciones al método. Por ejemplo, diferencias en la medición de productividad; los empleadores pueden tener una forma diferente de evaluar la productividad que la implementada en el modelo a través características observadas. Asimismo, las características observadas incluidas en el modelo también podrían verse afectadas por la "discriminación" y no ajusto las estimaciones para esta posible amenaza.

Además, intento controlar los problemas de autoselección en la fuerza laboral usando la corrección de dos pasos de Heckman mientras modelo la participación en la fuerza laboral en función de la edad, el estado civil y el número de hijos. Sin embargo, este enfoque solo aborda cuestiones de selección de muestras impulsadas por las características observadas.

Para investigar la posible triple penalización de género, maternidad e informalidad, realizo descomposiciones Oaxaca-Blinder con y sin corrección de Heckman para tres submuestras diferentes. Primero, una muestra que incluye a todos los hombres y mujeres, segundo, una muestra que incluye solo a madres y a los padres, y por último, una muestra que incluye solo a las madres y los padres del sector informal. Las características observadas controladas en los modelos son las mismas: edad, nivel educativo e industria en dos dígitos. El foco principal no está en las magnitudes de las brechas, sino en la dirección y la significancia estadística del efecto que la paternidad y la informalidad tienen sobre las diferencias salariales. Varios problemas potenciales, como el error de medición de salarios, podrían generar conclusiones engañosas sobre el tamaño de la brecha si analizáramos la magnitud de la penalidad. Sin embargo, asumí que no había diferencias sistemáticas en la medición de salarios entre las diferentes submuestras analizadas.

Para la descomposición sin la corrección de Heckman, utilizo los salarios a nivel para ver las diferencias brutas. Sin embargo, los modelos definitivos con correcciones de autoselección se han estimado utilizando el logaritmo de los salarios como variables de resultado.

Como se observa en todas las diferentes especificaciones, existe una diferencia salarial inexplicable. Para toda la muestra de hombres y mujeres en el modelo sin correcciones, la brecha salarial total es de alrededor del 44 por ciento, de las cuales las "dotaciones" de características individuales contribuyen a su reducción. En la muestra de madres y padres, la magnitud de la brecha salarial aumenta en comparación con toda la muestra y las dotaciones no contribuyen más a su reducción, lo que indica una posible penalización sobre la maternidad.

Además, el análisis de descomposición para madres y padres en el sector informal muestra una reducción de la brecha salarial en comparación con la muestra completa de madres y padres. Sin embargo, esta reducción se debe principalmente a un aumento en la contribución de las dotaciones. Por tanto, esto último contribuye a reducir el efecto negativo de estar en el sector informal.

El análisis de descomposición mediante correcciones de selección de muestras para hombres y mujeres arroja resultados similares, con una contribución de las dotaciones que reduce la brecha salarial. En el caso de la muestra de madres y padres, aún se observa una brecha salarial que aumenta con respecto a la brecha salarial de solo mujeres y hombres. A diferencia del modelo sin corrección, dotaciones como el nivel educativo y la industria contribuyen a reducir la brecha salarial entre padres y madres. En la tabla \ref{tab:oaxaca} se pueden encontrar más detalles de las descomposiciones Blinder-Oaxaca.
%
%-----------------------------------------------------------------
% Tab 1
%-----------------------------------------------------------------
\begin{table}[H]
    \singlespacing
	\small
	\centering 
	\begin{adjustbox}{max width=\textwidth}
		\begin{threeparttable}
			\caption{Descomposición Oaxaca-Blinder}
			\label{tab:oaxaca}
			\begin{tabular}{@{}l*{6}{c}@{}}
				\toprule
				 		        															& 
				\multicolumn{3}{c}{Descomposición Oaxaca-Blinder} 							&
				\multicolumn{3}{c}{Descomposición Oaxaca-Blinder con corrección Heckman }	\\ 
																							& 
				\multicolumn{3}{c}{Outcome: salario (córdobas)} 							&
				\multicolumn{3}{c}{Outcome: logaritmo del salario}							\\ \cmidrule(lr){2-4} \cmidrule(l){5-7}		
									&
				Hombre vs. Mujer	&
				Padre vs. Madre 	& 
				Padre vs. Madre 	& 
				Hombre vs. Mujer	&
				Padre vs. Madre 	& 
				Padre vs. Madre 	\\
									&
									&
									&
				Sector Informal		& 
									&
									&
				Sector Informal		\\ \cmidrule(lr){2-4} \cmidrule(l){5-7}	
									&
				(1) 				& 
				(2) 				& 
				(3) 				& 
				(4) 				& 
				(5) 				& 
				(6)					\\      
				\midrule
                \primitiveinput{tables/oaxaca.tex}	
				\bottomrule
			\end{tabular}
			\begin{tablenotes}
				\setlength\labelsep{0pt}
				\footnotesize
				\item \textit{Notas}: \textit{T}-statistics are reported in parentheses.
				\item \sym{***} Significant at the 1 percent level.
				\item \sym{**} Significant at the 5 percent level. 
				\item \sym{*} Significant at the 10 percent level.
			\end{tablenotes}
		\end{threeparttable}
	\end{adjustbox}
\end{table}
%
Los resultados de la descomposición de la muestra de padres y madres del sector informal muestran un aumento de la brecha salarial con respecto a la muestra total y la muestra de padres y madres únicamente. La contribución de las dotaciones impulsa los cambios después de controlar los problemas de selección. Por tanto, la informalidad implica una penalización adicional para las mujeres después de controlar la participación laboral.

Las características observadas generan los cambios en los modelos con y sin corrección por selección de la muestra. Por ejemplo, la presencia de mujeres y madres educadas en el sector informal sesga las estimaciones de la brecha salarial si no se controla autoselección. Por lo tanto, ser madre y estar en el sector informal representa penalidades adicionales en términos de diferencias salariales en comparación con sus pares masculinos y padres.

%-----------------------------------------------------------------
\subsection{Matrices de transición de cadena de Markov}
%-----------------------------------------------------------------
La pregunta que surge es si esas mujeres y madres se auto-seleccionan en este sector informal o si es resultado de la falta de trabajos formales. Aunque esta pregunta requiere una investigación más exhaustiva y estrategias de identificación avanzadas, tengo la intención de proporcionar información sobre los patrones de movilidad de las mujeres y las madres en comparación con sus pares en diferentes situaciones laborales.

Las probabilidades de transición de las matrices de Markov pretenden aproximar la movilidad y absorción de diferentes estados laborales como la voluntad o el efecto residual de las decisiones del mercado laboral. Por ejemplo, observar una alta movilidad entre las mujeres que inicialmente ingresaron a un trabajo formal pero luego pasaron a un trabajo informal o al desempleo o inactividad podría indicar la disposición y el deseo de trabajos y horarios más flexibles. Sin embargo, estos resultados no son causales y son simplemente el reflejo de los patrones de movilidad de los individuos en la muestra. En la tabla \ref{tab:matrices} se pueden encontrar más detalles de las matrices de transición de la cadena de Markov.
%-----------------------------------------------------------------
\subsection{Transiciones por sexo}
%-----------------------------------------------------------------
Las mujeres inicialmente observadas en trabajos informales tienen un 58 por ciento de probabilidad de permanecer en este estado, un 15 por ciento de convertirse en trabajadoras informales por cuenta propia y un 16 por ciento de dejar la fuerza laboral. Los hombres presentan patrones similares, pero en lugar de salir de la fuerza laboral, el 25 por ciento pasa de asalariados informales a cuenta propia informal.

Por otro lado, los trabajadores asalariados formales parecen ser un estado absorbente para las mujeres; una vez que ingresan a un trabajo formal, el 91 por ciento de ellas permanece en estos trabajos, solo el 3 por ciento pasa a trabajadores asalariados informales. Los hombres, por el contrario, una vez que ingresan a un puesto asalariado formal, el 6 por ciento de ellos pasa al trabajador asalariado informal y el 89 permanece como formal. De manera similar, el trabajador informal por cuenta propia parece ser un estatus absorbente para hombres y mujeres; El 78 de los hombres y el 77 por ciento de las mujeres que ingresan a esos trabajos permanecen en ellos.

Asimismo, el desempleo y los inactivos o fuera de la fuerza laboral tienen una probabilidad de permanencia similar para hombres y mujeres; sin embargo, el tipo de trabajo a los que transitan posteriormente difiere. La mayoría de los hombres que estaban desempleados o inactivos pasaron a ser trabajadores asalariados informales, mientras que para las mujeres, el 17 por ciento de los desempleadas y el 20 por ciento de los inactivas pasaron a ser cuenta propista informal. Este resultado podría estar en línea con las diferencias observadas en el sector informal evidenciadas en la descomposición Blinder-Oaxaca.
%-----------------------------------------------------------------
\subsection{Transiciones por paternidad}
%-----------------------------------------------------------------
Para las madres, se observan patrones similares en el estatus formal asalariado, el 91 por ciento de las madres inicialmente observadas en este tipo de trabajo permanecieron allí. En el caso de las trabajadoras asalariados informales, el 19 por ciento pasó a ser cuenta propista informal y solo el 16 por ciento pasó a estar desempleada o inactiva, cifra menor que el 21 por ciento observado para toda la muestra.

Curiosamente, la probabilidad de permanecer desempleadas o inactivas se reduce para las madres en comparación con las mujeres no madres, con un aumento en las transiciones hacia asalariados informales, pero principalmente hacia trabajadores informales por cuenta propia. Los padres también reducen sus probabilidades de permanecer desempleados o inactivos, aumentando ahora las transiciones hacia trabajadores por cuenta propia.

Los trabajos formales para las mujeres y sobre todo para las madres parecen ser el estado preferido y absorbente. Asimismo, las madres salen del desempleo y la inactividad con mayor frecuencia en comparación con las transiciones de las mujeres al usar toda la muestra.
%
%-----------------------------------------------------------------
% Tab 3
%-----------------------------------------------------------------
\begin{table}[H]
    \singlespacing
	\small
	\centering 
	\begin{adjustbox}{max width=\textwidth}
		\begin{threeparttable}
			\caption{Matrices de transición de la cadena de Markov para mujeres y hombres y padres y madres}
			\label{tab:matrices}
			\begin{tabular}{@{}l*{6}{c}@{}}
				\toprule
									&
				\multicolumn{2}{c}{Asalariado}		&
				\multicolumn{2}{c}{Cuentapropista}	&
				Desempleo			&
				Fuera de la 		\\
									&
				Informal			&
				Formal				&
				Informal			&
				Formal				&
									&
				Fuerza Laboral		\\ \cmidrule(lr){2-3} \cmidrule(lr){4-5}
				Clasificación de ocupación		&
				(1) 				& 
				(2) 				& 
				(3) 				& 
				(4) 				& 
				(5) 				& 
				(6) 				\\
				\midrule				
				$Periodo~t$ 				&
				\multicolumn{6}{c}{$Periodo~t + 1$} \\
				\cmidrule{2-7}
				\multicolumn{7}{@{}l}{\textit{Panel A. Hombres}}			\\
				\primitiveinput{tables/markow_sex_0.tex}	
				[0.5em]
				\multicolumn{7}{@{}l}{\textit{Panel B. Mujeres}}			\\				
                \primitiveinput{tables/markow_sex_1.tex}	
				[0.5em]
				\multicolumn{7}{@{}l}{\textit{Panel C. Padres}}				\\
				\primitiveinput{tables/markow_motandfat_0.tex}	
				[0.5em]
				\multicolumn{7}{@{}l}{\textit{Panel D. Madres}}				\\				
                \primitiveinput{tables/markow_motandfat_1.tex}					
				\bottomrule
			\end{tabular}
			\begin{tablenotes}
				\setlength\labelsep{0pt}
				\footnotesize
				\item \textit{Notas}: 
			\end{tablenotes}
		\end{threeparttable}
	\end{adjustbox}
\end{table}
%

%-----------------------------------------------------------------
% CONCLUSIONS AND POLICY RECOMMENDATIONS
%-----------------------------------------------------------------
%-----------------------------------------------------------------
\section{Conclusions and policy recommendation}
%-----------------------------------------------------------------
En este artículo, reviso la literatura existente sobre las brechas salariales de género, sus tendencias y posibles explicaciones, centrándome en la investigación reciente que ha tenido como objetivo el estudio de la contribución de los componentes no explicados de las diferencias salariales entre hombres y mujeres. De ahí que se hiciera especial énfasis en la brecha familiar y la informalidad. Este estudio también pretendía aportar evidencia de la existencia de una brecha salarial de género en Nicaragua y la triple penalización que podría enfrentar una madre en un país con altos índices de informalidad. Para ello, ha llevado a cabo metodologías usualmente utilizada por la literatura en este tipo de análisis con la intención de proveer evidencia sobre el impacto del género en los ingresos, las diferencias salariales entre diferentes grupos, el impacto de la maternidad, así como información acerca de la posible selección en trabajos flexibles por parte de las mujeres y madres.

La revisión de la literatura más relevante y reciente muestra, por un lado, que los factores tradicionales, especialmente los relacionados con el capital humano, explican muy poco de las brechas de ingresos actuales. Por otro lado, revela que la investigación sobre el impacto de aspectos como la experiencia laboral y la participación en el mercado laboral presentan resultados mixtos según las condiciones del mercado laboral y otras características. Asimismo, la evidencia empírica en el campo de las brechas familiares también es controversial debido a potenciales problemas de endogeneidad. Sin embargo, recientemente se han generado investigaciones rigurosas que han demostrado el impacto de la maternidad en la desigualdad de género. El enfoque principal de esta investigación ha sido en las economías desarrolladas con estados de bienestar y condiciones del mercado laboral muy particulares.

La estimación de ecuaciones salariales permitió identificar el efecto del género sobre los ingresos y variables adicionales que tienen roles esenciales en la determinación de los salarios de mujeres y hombres en Nicaragua. Si bien la paternidad y el matrimonio tienen un impacto positivo en los ingresos de los hombres, la evidencia provista indica que la maternidad y el matrimonio probablemente tengan efectos adversos en los salarios de las mujeres. El número de hijos no afectó la penalidad por maternidad; por lo tanto, podríamos sospechar que los efectos en los ingresos comienzan desde el primer hijo y se mantienen constantes independientemente del número de hijos. Del mismo modo, las interrupciones del trabajo en el trimestre anterior parecen tener un impacto negativo en los salarios de las mujeres y no para los hombres.

El análisis de descomposición mostró el efecto del género, la paternidad, el estado de formalización y el papel de las características no observadas de las mujeres cuando participan en la fuerza laboral. La evidencia presentada sugiere la presencia de una brecha salarial de género, incluso después de corregir la autoselección en la fuerza laboral. Además, las estimaciones de descomposición en diferentes submuestras, en este caso entre hombres y mujeres en su conjunto y luego entre padres y madres, indican un aumento en la brecha salarial dada la condición de paternidad. Por lo tanto, la evidencia presentada apunta a la existencia de una penalización por maternidad, que aumenta la brecha que enfrentan las mujeres durante su vida laboral en comparación con sus pares masculinos y mujeres sin hijos.

Además, después de corregir por autoselección, las descomposiciones de Blinder-Oaxaca proporcionaron evidencia del impacto de la informalidad y las características no observadas de las madres que participan en dicho sector. El modelo final sugiere una penalización adicional impuesta por el sector informal que contrapesa las características de capital humano de las madres que participan en el sector informal.

El análisis presentado por las matrices de transición sugiere que los trabajos asalariados formales son un estado deseable y absorbente para mujeres y madres. Las mujeres y las madres presentaron probabilidades de permanencia incluso más altas que los hombres y más movilidad fuera del desempleo y la inactividad. Indicando por ende que los trabajos informales o más flexibles no son más deseados por las mujeres y las madres, sino que son el resultado de la falta de mejores trabajos en el mercado laboral. Sin embargo, también se reconoce la necesidad de proveer pruebas adicionales, con énfasis en causalidad, para respaldar dicha afirmación.

Aunque este estudio no tuvo como objetivo analizar la naturaleza de la discriminación a la cual se enfrentan las mujeres, dadas las características del mercado laboral y la movilidad del mercado laboral, es probable que la discriminación estadística esté presente en este contexto. Por lo tanto, en un posible escenario de comportamiento discriminatorio, legislación de protección del empleo contra la discriminación de género podría desempeñar un papel fundamental en la reducción de la persistencia de la brecha salarial de género. Un paso importante sería mejorar el marco legal con una revisión amplia y participativa de la legislación exitosa implementada en países con condiciones similares. Entre esta legislación, es meritorio un énfasis particular en la licencia parental, su carga para el sector privado y sus efectos sobre las decisiones laborales y la vinculación de las mujeres al mercado laboral. Se necesitan más investigaciones para identificar las fuentes de discriminación e informar a los responsables de la formulación de políticas.

Finalmente, con respecto a un posible sesgo de género causado por las decisiones de asignación de tiempo para el trabajo reproductivo, las políticas familiares adecuadas, como el cuidado de los hijos y la licencia parental, son fundamentales para contrarrestar las desventajas de las mujeres en el mercado formal. Las mejoras cualitativas en la oferta de cuidado infantil y la extensión de la licencia parental para los padres constituyen opciones potenciales para reducir las desventajas de las mujeres y la discriminación del empleador.
%-----------------------------------------------------------------
% REFERECNES
%-----------------------------------------------------------------
\newpage
%\nocite{*}
\printbibliography

%%-----------------------------------------------------------------
\section*{Referencias}
%-----------------------------------------------------------------
\begingroup
\noindent
\setlength{\parindent}{-0.5in}
\setlength{\leftskip}{0.5in}
%\setlength{\parskip}{10pt}

Akerlof, G. A., \& Kranton, R. E. (2000). Economics and Identity*. Quarterly Journal of Economics, 115(3), 715–753. \url{https://doi.org/10.1162/003355300554881}

Akerlof, G. A., \& Kranton, R. E. (2002). Identity and Schooling: Some Lessons for the Economics of Education. Journal of Economic Literature, 40(4), 1167–1201. \url{https://doi.org/10.1257/002205102762203585}

Akerlof, G. A., \& Kranton, R. E. (2005). Identity and the economics of organizations. Journal of Economic Perspectives, 19(1), 9–32. \url{https://doi.org/10.1257/0895330053147930}

Altonji, J. G., \& Blank, R. M. (1999a). Chapter 48 Race and gender in the labor market. Handbook of Labor Economics. \url{https://doi.org/10.1016/S1573-4463(99)30039-0}

Arcidiacono, P. (2004). Ability sorting and the returns to college major. Journal of Econometrics, 121(1–2), 343–375. \url{https://doi.org/10.1016/J.JECONOM.2003.10.010}

Arrow, K. J. (1973). The theory of discrimination. In discrimination in labor markets (pp. 3–33). Princeton University Press, Princeton. \url{https://doi.org/10.1515/9781400867066-003}

Becker, G. (1985). Human Capital, Effort, and the Sexual Division of Labor. Journal of Labor Economics, 3(1), S33-58. Retrieved from \url{https://econpapers.repec.org/article/ucpjlabec/v\_3a3\_3ay\_3a1985\_3ai\_3a1\_3ap\_3as33-58.html}

Becker, G. S. (1957). The Economics of Discrimination. University of Chicago Press, 18(3), 276. \url{https://doi.org/10.2307/3708738}

Becker, G. S., Hubbard, W. H. J., \& Murphy, K. M. (2010). The Market for College Graduates and the Worldwide Boom in Higher Education of Women. American Economic Review, 100(2), 229–233. \url{https://doi.org/10.1257/aer.100.2.229}

Ben Yahmed, S. (2018). Formal but Less Equal. Gender Wage Gaps in Formal and Informal Jobs in Urban Brazil. World Development, 101, 73–87. \url{https://doi.org/10.1016/j.worlddev.2017.08.012}

Benard, S., \& Correll, S. J. (2010). Normative discrimination and the motherhood penalty. Gender and Society, 24(5), 616–646. \url{https://doi.org/10.1177/0891243210383142}

Berniell, L., Mata, D. De, Edo, M., \& Marchionni, M. (2019). Gender Gaps in Labor Informality : The Motherhood Effect. Documentos de Trabajo Del CEDLAS, no. 247(Issue 247), 1–41. Retrieved from \url{http://sedici.unlp.edu.ar/handle/10915/77486}

Bertrand, M. (2011). New perspectives on gender. Handbook of Labor Economics (Vol. 4). Elsevier. \url{https://doi.org/10.1016/S0169-7218(11)02415-4}

Bian, L., Leslie, S.-J., \& Cimpian, A. (2017). Gender stereotypes about intellectual ability emerge early and influence children's interests. Science (New York, N.Y.), 355(6323), 389–391. \url{https://doi.org/10.1126/science.aah6524}

Blau, F. D., \& Kahn, L. (1995). The Gender Earnings Gap: Some International Evidence. NBER Chapters. Cambridge, MA. \url{https://doi.org/10.3386/w4224}

Blau, F. D., \& Kahn, L. M. (2000). Gender Differences in Pay. Journal of Economic Perspectives, 14(4), 75–100. \url{https://doi.org/10.1257/jep.14.4.75}

Blau, F. D., \& Kahn, L. M. (2007). The Gender Pay Gap. Academy of Management Perspectives, 21(1), 7–23. \url{https://doi.org/10.5465/amp.2007.24286161}

Blau, F. D., \& Kahn, L. M. (2017). The Gender Wage Gap: Extent, Trends, and Explanations. Journal of Economic Literature, 55(3), 789–865. \url{https://doi.org/10.1257/jel.20160995}

Blinder, A. S. (1973). Wage Discrimination: Reduced Form and Structural Estimates. The Journal of Human Resources, 8(4), 436. \url{https://doi.org/10.2307/144855}

Borjas, G. J. (2016). Labor Economics. McGraw-hill (Vol. 7).

Borrowman, M., \& Klasen, S. (2019). Drivers of Gendered Sectoral and Occupational Segregation in Developing Countries. Feminist Economics, 1–33. \url{https://doi.org/10.1080/13545701.2019.1649708}

Bosch, M., \& Maloney, W. F. (2010). Comparative analysis of labor market dynamics using Markov processes: An application to informality. Labour Economics, 17(4), 621–631. \url{https://doi.org/10.1016/j.labeco.2010.01.005}

Ceci, S. J., Ginther, D. K., Kahn, S., \& Williams, W. M. (2014). Women in academic science: A changing landscape. Psychological Science in the Public Interest, Supplement, 15(3), 75–141. \url{https://doi.org/10.1177/1529100614541236}

Cha, Y., \& Weeden, K. A. (2014). Overwork and the Slow Convergence in the Gender Gap in Wages. American Sociological Review, 79(3), 457–484. \url{https://doi.org/10.1177/0003122414528936}

Charles, K. K., Guryan, J., \& Pan, J. (2018). The Effects of Sexism on American Women: The Role of Norms vs. Discrimination. SSRN Electronic Journal. \url{https://doi.org/10.2139/ssrn.3233788}

Charness, G., \& Gneezy, U. (2012). Strong Evidence for Gender Differences in Risk Taking. Journal of Economic Behavior and Organization, 83(1), 50–58. \url{https://doi.org/10.1016/j.jebo.2011.06.007}

Clark, S., Kabiru, C. W., Laszlo, S., \& Muthuri, S. (2019). The Impact of Childcare on Poor Urban Women's Economic Empowerment in Africa. Demography, 56(4), 1247–1272. \url{https://doi.org/10.1007/s13524-019-00793-3}

Claudia Goldin. (2008). Gender Gap. Retrieved November 15, 2019, from \url{https://www.econlib.org/library/Enc1/GenderGap.html}

Correll, S. J., Benard, S., \& Paik, I. (2007). Getting a job: Is there a motherhood penalty? American Journal of Sociology, 112(5), 1297–1338. \url{https://doi.org/10.1086/511799}

Croson, R., \& Gneezy, U. (2009). Gender Differences in Preferences. Journal of Economic Literature, 47(2), 448–474. \url{https://doi.org/10.1257/jel.47.2.448}

Del Boca, D., Flinn, C., \& Wiswall, M. (2014). Household choices and child development. Review of Economic Studies, 81(1), 137–185. \url{https://doi.org/10.1093/restud/rdt026}

ECLAC. (2016). Women: The Most Harmed by Unemployment. Notes for Equality (Vol. 22). Retrieved from \url{http://www.cepal.org/en/publications/type/preliminary-overview-economies-latin-america-and-caribbean}

England, M. J. B. and P. (2001). The Wage Penalty for Motherhood. American Sociological Review, 53(9), 1689–1699. \url{https://doi.org/10.1017/CBO9781107415324.004}

England, P., Farkas, G., Kilbourne, B. S., \& Dou, T. (1988). Explaining Occupational Sex Segregation and Wages: Findings from a Model with Fixed Effects. American Sociological Review, 53(4), 544.\url{https://doi.org/10.2307/2095848}

Fortin, N., Lemieux, T., \& Firpo, S. (2011). Decomposition Methods in Economics. Handbook of Labor Economics (Vol. 4). \url{https://doi.org/10.1016/S0169-7218(11)00407-2}

Fortin, N. M. (2005). Gender role attitudes and the labour-market outcomes of women across OECD countries. Oxford Review of Economic Policy, 21(3), 416–438. \url{https://doi.org/10.1093/oxrep/gri024}

Fuller, W. C., Manski, C. F., \& Wise, D. A. (1982). New Evidence on the Economic Determinants of Postsecondary Schooling Choices. The Journal of Human Resources, 17(4), 477. \url{https://doi.org/10.2307/145612}

Gaddis, I., \& Klasen, S. (2014). Economic development, structural change, and women's labor force participation:: A reexamination of the feminization U hypothesis. Journal of Population Economics, 27(3), 639–681. \url{https://doi.org/10.1007/s00148-013-0488-2}

Gneezy, U., Leonard, K. L., \& List, J. A. (2009). Gender Differences in Competition: Evidence From a Matrilineal and a Patriarchal Society. Econometrica, 77(5), 1637–1664. \url{https://doi.org/10.3982/ecta6690}

Goldin, C. (2006a). The Quiet Revolution That Transformed Women's Employment, Education, and Family. American Economic Review, 96(2), 1–21. \url{https://doi.org/10.1257/000282806777212350}

Goldin, C. (2006b). The quiet revolution that transformed women's employment, education, and family. American Economic Review, 96(2), 1–21. \url{https://doi.org/10.1257/000282806777212350}

Goldin, C. (2014). A grand gender convergence: Its last chapter. American Economic Review, 104(4), 1091–1119. \url{https://doi.org/10.1257/aer.104.4.1091}

Goldin, C., \& Katz, L. F. (2002). The power of the pill: Oral contraceptives and women's career and marriage decisions. Journal of Political Economy, 110(4), 730–770. \url{https://doi.org/10.1086/340778}

González, K. P., Stoner, C., \& Jovel, J. E. (2003). Examining the Role of Social Capital in Access to College for Latinas: Toward a College Opportunity Framework. Journal of Hispanic Higher Education, 2(2), 146–170. \url{https://doi.org/10.1177/1538192702250620}

Heckman, J. J. (1979). Sample Selection Bias as a Specification Error. Econometrica, 47(1), 153. \url{https://doi.org/10.2307/1912352}

Hersch, J., \& Stratton, L. S. (2002). Housework and wages. Journal of Human Resources, 37(1), 217–229. \url{https://doi.org/10.2307/3069609}

ILO. (2019). A quantum leap for gender equality : for a better future of work for all. Geneva. Retrieved from \url{www.ilo.org/publns}.

Jann, B. (2008). The Blinder-Oaxaca decomposition for linear regression models. Stata Journal, 8(4), 453–479. \url{https://doi.org/10.1177/1536867x0800800402}

Klasen, S., \& Pieters, J. (2015). What explains the stagnation of female labor force participation in Urban India? World Bank Economic Review, 29(3), 449–478. \url{https://doi.org/10.1093/wber/lhv003}

Kleven, H., Landais, C., \& Søgaard, J. E. (2019). Children and Gender Inequality: Evidence from Denmark. American Economic Journal: Applied Economics, 11(4), 181–209. \url{https://doi.org/10.1257/app.20180010}

Macpherson, D. A., \& Hirsch, B. T. (1995). Wages and Gender Composition: Why do Women's Jobs Pay Less? Journal of Labor Economics, 13(3), 426–471. \url{https://doi.org/10.1086/298381}

Mincer, Jacob and Polachek, S. (1974). Family Investments in Human Capital: Earnings of Women: Comment. Journal of Political Economy, 82(2, Part 2), S109–S110. \url{https://doi.org/10.1086/260294}

Monroy, E. (2008). Equidad de Genero en el Mercado Laboral para Nicaragua. Serie de Cuadernos de Género Para …, 8(2), 1307–1319. Retrieved from \url{http://web.worldbank.org/archive/website01404/WEB/IMAGES/CGANICAR.PDF}

Niederle, M., \& Vesterlund, L. (2007, August 1). Do women shy away from competition? Do men compete too much? Quarterly Journal of Economics. Narnia. \url{https://doi.org/10.1162/qjec.122.3.1067}

Nordman, C. J., Rakotomanana, F., \& Roubaud, F. (2016). Informal versus Formal: A Panel Data Analysis of Earnings Gaps in Madagascar. World Development, 86, 1–17. \url{https://doi.org/10.1016/j.worlddev.2016.05.006}

Oaxaca, R. (1973). Male-Female Wage Differentials in Urban Labor Markets. International Economic Review, 14(3), 693. \url{https://doi.org/10.2307/2525981}

Phelps, E. S. (1972). The Statistical theory of Racism and Sexism. American Economic Review, 62(4), 659–661. \url{https://doi.org/10.2307/1806107}

Reskin, B. (1993). Sex segregation in the workplace. Annual Review of Sociology. Vol. 19, 241–270. \url{https://doi.org/10.1146/annurev.soc.19.1.241}

Ruhm, C. J. (1998). The Economic Consequences of Parental Leave Mandates: Lessons from Europe. The Quarterly Journal of Economics, 113(1), 285–317. \url{https://doi.org/10.1162/003355398555586}

Ruzik, A., \& Rokicka, M. (2012). The Gender Pay Gap in Informal Employment in Poland. SSRN Electronic Journal. \url{https://doi.org/10.2139/ssrn.1674939}

Sahoo, S., \& Klasen, S. (2018). Gender Segregation in Education and Its Implications for Labour Market Outcomes: Evidence from India. Educational Research Journal, (11660), 1–52. Retrieved from \url{https://www.iza.org/publications/dp/11660/gender-segregation-in-education-and-its-implications-for-labour-market-outcomes-evidence-from-india}

Schneeweis, N., \& Zweimüller, M. (2012). Girls, girls, girls: Gender composition and female school choice. Economics of Education Review, 31(4), 482–500. \url{https://doi.org/10.1016/j.econedurev.2011.11.002}

Sigle-Rushton, W., \& Waldfogel, J. (2007). The incomes of families with children: A cross-national comparison. Journal of European Social Policy, 17(4), 299–318. \url{https://doi.org/10.1177/0958928707082474}

Sorensen, E. (1990). The Crowding Hypothesis and Comparable Worth. The Journal of Human Resources, 25(1), 55. \url{https://doi.org/10.2307/145727}

Tansel, A. (2005). Wage Earners, Self-Employed and Gender in the Informal Sector in Turkey. SSRN Electronic Journal. \url{https://doi.org/10.2139/ssrn.263275}

The International Labour Organization. (2018). Global wage report 2018 / 19 what lies behind gender pay gaps. Retrieved from \url{https://www.ilo.org/wcmsp5/groups/public/---dgreports/---dcomm/---publ/documents/publication/wcms\_650553.pdf}

UNDP. (2014). El mercado laboral de Nicaragua desde un enfoque de género.

Waldfogel, J. (1998). Understanding the "Family Gap" in Pay for Women with Children. Journal of Economic Perspectives, 12(1), 137–156. \url{https://doi.org/10.1257/jep.12.1.137}

Weichselbaumer, D., \& Winter-Ebmer, R. (2005). A meta-analysis of the international gender wage gap. Journal of Economic Surveys, 19(3), 479–511. \url{https://doi.org/10.1111/j.0950-0804.2005.00256.x}

Williams, W. M., \& Ceci, S. J. (2015). National hiring experiments reveal 2:1 faculty preference for women on STEM tenure track. Proceedings of the National Academy of Sciences of the United States of America, 112(17), 5360–5365. \url{https://doi.org/10.1073/pnas.1418878112}

World Bank. (2011). World Development Report 2012.

World Economic Forum. (2016). The global gender gap report 2016 Insight Report. World Economic Forum (Vol. 25). \url{https://doi.org/10.1177/0192513X04267098}

\endgroup
%-----------------------------------------------------------------
% APPENDIX
%-----------------------------------------------------------------
%\newpage 

%\begin{appendices}	
%	\section{Tablas}
%\end{appendices}
%-----------------------------------------------------------------
% END ------------------------------------------------------------
%-----------------------------------------------------------------
\end{document}
%-----------------------------------------------------------------