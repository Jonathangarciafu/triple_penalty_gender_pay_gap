%-----------------------------------------------------------------
\section{Conclusions and policy recommendation}
%-----------------------------------------------------------------
En este artículo, reviso la literatura existente sobre las brechas salariales de género, sus tendencias y posibles explicaciones, centrándome en la investigación reciente que ha tenido como objetivo el estudio de la contribución de los componentes no explicados de las diferencias salariales entre hombres y mujeres. De ahí que se hiciera especial énfasis en la brecha familiar y la informalidad. Este estudio también pretendía aportar evidencia de la existencia de una brecha salarial de género en Nicaragua y la triple penalización que podría enfrentar una madre en un país con altos índices de informalidad. Para ello, ha llevado a cabo metodologías usualmente utilizada por la literatura en este tipo de análisis con la intención de proveer evidencia sobre el impacto del género en los ingresos, las diferencias salariales entre diferentes grupos, el impacto de la maternidad, así como información acerca de la posible selección en trabajos flexibles por parte de las mujeres y madres.

La revisión de la literatura más relevante y reciente muestra, por un lado, que los factores tradicionales, especialmente los relacionados con el capital humano, explican muy poco de las brechas de ingresos actuales. Por otro lado, revela que la investigación sobre el impacto de aspectos como la experiencia laboral y la participación en el mercado laboral presentan resultados mixtos según las condiciones del mercado laboral y otras características. Asimismo, la evidencia empírica en el campo de las brechas familiares también es controversial debido a potenciales problemas de endogeneidad. Sin embargo, recientemente se han generado investigaciones rigurosas que han demostrado el impacto de la maternidad en la desigualdad de género. El enfoque principal de esta investigación ha sido en las economías desarrolladas con estados de bienestar y condiciones del mercado laboral muy particulares.

La estimación de ecuaciones salariales permitió identificar el efecto del género sobre los ingresos y variables adicionales que tienen roles esenciales en la determinación de los salarios de mujeres y hombres en Nicaragua. Si bien la paternidad y el matrimonio tienen un impacto positivo en los ingresos de los hombres, la evidencia provista indica que la maternidad y el matrimonio probablemente tengan efectos adversos en los salarios de las mujeres. El número de hijos no afectó la penalidad por maternidad; por lo tanto, podríamos sospechar que los efectos en los ingresos comienzan desde el primer hijo y se mantienen constantes independientemente del número de hijos. Del mismo modo, las interrupciones del trabajo en el trimestre anterior parecen tener un impacto negativo en los salarios de las mujeres y no para los hombres.

El análisis de descomposición mostró el efecto del género, la paternidad, el estado de formalización y el papel de las características no observadas de las mujeres cuando participan en la fuerza laboral. La evidencia presentada sugiere la presencia de una brecha salarial de género, incluso después de corregir la autoselección en la fuerza laboral. Además, las estimaciones de descomposición en diferentes submuestras, en este caso entre hombres y mujeres en su conjunto y luego entre padres y madres, indican un aumento en la brecha salarial dada la condición de paternidad. Por lo tanto, la evidencia presentada apunta a la existencia de una penalización por maternidad, que aumenta la brecha que enfrentan las mujeres durante su vida laboral en comparación con sus pares masculinos y mujeres sin hijos.

Además, después de corregir por autoselección, las descomposiciones de Blinder-Oaxaca proporcionaron evidencia del impacto de la informalidad y las características no observadas de las madres que participan en dicho sector. El modelo final sugiere una penalización adicional impuesta por el sector informal que contrapesa las características de capital humano de las madres que participan en el sector informal.

El análisis presentado por las matrices de transición sugiere que los trabajos asalariados formales son un estado deseable y absorbente para mujeres y madres. Las mujeres y las madres presentaron probabilidades de permanencia incluso más altas que los hombres y más movilidad fuera del desempleo y la inactividad. Indicando por ende que los trabajos informales o más flexibles no son más deseados por las mujeres y las madres, sino que son el resultado de la falta de mejores trabajos en el mercado laboral. Sin embargo, también se reconoce la necesidad de proveer pruebas adicionales, con énfasis en causalidad, para respaldar dicha afirmación.

Aunque este estudio no tuvo como objetivo analizar la naturaleza de la discriminación a la cual se enfrentan las mujeres, dadas las características del mercado laboral y la movilidad del mercado laboral, es probable que la discriminación estadística esté presente en este contexto. Por lo tanto, en un posible escenario de comportamiento discriminatorio, legislación de protección del empleo contra la discriminación de género podría desempeñar un papel fundamental en la reducción de la persistencia de la brecha salarial de género. Un paso importante sería mejorar el marco legal con una revisión amplia y participativa de la legislación exitosa implementada en países con condiciones similares. Entre esta legislación, es meritorio un énfasis particular en la licencia parental, su carga para el sector privado y sus efectos sobre las decisiones laborales y la vinculación de las mujeres al mercado laboral. Se necesitan más investigaciones para identificar las fuentes de discriminación e informar a los responsables de la formulación de políticas.

Finalmente, con respecto a un posible sesgo de género causado por las decisiones de asignación de tiempo para el trabajo reproductivo, las políticas familiares adecuadas, como el cuidado de los hijos y la licencia parental, son fundamentales para contrarrestar las desventajas de las mujeres en el mercado formal. Las mejoras cualitativas en la oferta de cuidado infantil y la extensión de la licencia parental para los padres constituyen opciones potenciales para reducir las desventajas de las mujeres y la discriminación del empleador.