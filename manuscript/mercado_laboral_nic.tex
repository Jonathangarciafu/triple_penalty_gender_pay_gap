%-----------------------------------------------------------------
\section{Mercado de trabajo en nicaragua -- Un Enfoque de Género}
%-----------------------------------------------------------------
En esta sección se brinda una breve descripción del mercado laboral nicaragüense con base en la Encuesta Continua de Hogares (ECH) del último trimestre de 2012. Todas las estadísticas descriptivas utilizan factores de expansión para facilitar la inferencia sobre la población. Para obtener una descripción más detallada, consulte la tabla 1 del apéndice.

La muestra total de la encuesta se divide en 49 por ciento de hombres y 51 por ciento de mujeres. Entre los hombres de la encuesta, el 63 por ciento pertenece a la población en edad de trabajar y entre las mujeres, esta proporción representa el 65 por ciento. La población ocupada, siguiendo la definición de la 19 CIET, asciende al 90 por ciento de los hombres y al 71 por ciento de las mujeres en la población en edad de trabajar. A pesar de las diferencias en la participación, las tasas de informalidad entre ambos grupos son similares, 80 por ciento para hombres y 81 por ciento para mujeres. De manera similar, la proporción de la población que vive en el sector rural representa el 44 por ciento de los hombres y el 41 por ciento de las mujeres.

En promedio, el ingreso mensual observado para los hombres es de 3,346 córdobas, lo que representa alrededor de US\$142 utilizando tipo de cambio al momento de la encuesta. Para las mujeres, el ingreso promedio es de 2,824 en moneda local, lo que corresponde a US\$120. Por tanto, los ingresos de las mujeres representan el 84 por ciento de los ingresos de los hombres. Sin embargo, el promedio de horas que trabaja una mujer (30 horas) representa el 75 por ciento del promedio de horas trabajadas por un hombre (41 horas).

Hombres y mujeres presentan niveles similares de escolaridad en cuanto a la finalización de los grados de primaria, secundaria y superior. La proporción de hombres sin educación formal es del 24 por ciento en comparación con el 23 por ciento de las mujeres. Asimismo, el 11 por ciento de los hombres completó la escuela primaria en comparación con el 10 por ciento de las mujeres. Por el contrario, el ocho por ciento de las mujeres completó la escuela secundaria en comparación con el siete por ciento de los hombres y el 10 por ciento de las mujeres logró un título avanzado en comparación con el nueve por ciento de los hombres. Por tanto, la diferencia parece no ser significativa; esto podría indicar que cualquier brecha salarial no se atribuye a diferencias educativas.

En términos de segregación ocupacional, las principales diferencias se observan en la industria agrícola y pesquera, donde el 45 por ciento de los hombres están empleados en comparación con solo el 21 por ciento de las mujeres. De manera similar, el sector de la construcción emplea al siete por ciento de la fuerza laboral masculina y solo al uno por ciento de la fuerza laboral femenina. En contraste, la industria del comercio, hoteles y restaurantes emplea al 38 por ciento de la fuerza laboral femenina y sólo al 19 por ciento de la fuerza laboral masculina. Además, la industria de la educación, la salud y la protección social contrata al 15\% de la fuerza laboral femenina y al 6\% de la masculina.

Además, los datos de la encuesta muestran a nivel del sector macroeconómico que aproximadamente el 66 por ciento de las mujeres se insertan en el sector terciario, especialmente en el comercio, ventas y servicios personales, y generalmente como trabajadoras por cuenta propia, mientras que este sector solo emplea al 38 por ciento. de la fuerza laboral masculina. Por el contrario, el sector primario contiene el 45 por ciento de la fuerza laboral masculina y el 21 por ciento de la fuerza laboral femenina. El sector secundario muestra una participación más equilibrada, empleando al 17 por ciento de los hombres y al 13 por ciento de las mujeres en la fuerza laboral.
